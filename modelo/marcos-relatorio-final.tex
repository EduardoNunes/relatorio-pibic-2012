\documentclass[11pt]{article}
\usepackage{relatorio}
\newcommand{\eng}[1]{\textit{#1}}
\usepackage[utf8x]{inputenc}

\begin{document}

\graphicspath{{figs/}}


\cabecalho{Relatório Final}

\dadosRelatorioFinal
{Criação de uma codificação musical a partir da análise de obras da literatura}
{Desenvolvimento e implementação de uma codificação para definir
  estruturas musicais}
{Marcos da Silva Sampaio}
{Pedro Ribeiro Kroger Junior}
{Genos}
{Informática em música, representação musical, teoria musical}
{AGOSTO DE 2005 A JULHO DE 2006}

\resumo{Diversas formas de codificação musical --- representação de
  elementos sonoros --- têm sido usadas desde o princípio da
  transcrição dos sons. O projeto do orientador visa desenvolver uma
  codificação que permita descrever estruturas musicais em níveis mais
  altos de abstração além de nota e duração, trabalho que pode vir a
  preencher uma lacuna da área de computação musical. Para o
  desenvolvimento desta codificação são necessárias soluções que
  definam sintaticamente estruturas musicais. O principal objetivo do
  plano do bolsista é estudar os problemas relativos à codificação
  musical e auxiliar nas citadas soluções. Para tanto houve um
  treinamento em importantes ferramentas, análise de estruturas
  musicais contidas em um conjunto de obras, e discussões do andamento
  do trabalho com o grupo de pesquisa. Esta pesquisa teve como
  resultado uma sintaxe para definir a forma, vozes e harmonia de um
  grupo de peças corais de Johann Sebastian Bach, e uma sintaxe para
  definir sujeitos, respostas e motivos de um grupo de fugas de J.S.
  Bach. Tivemos ainda a compreensão dos principais problemas
  referentes a \eng{Music Information Retrival --- MIR}, e aprendizado
  do uso de ferramentas e software livre. Um dos possíveis trabalhos
  futuros consequentes da atual pesquisa é a criação de um banco de
  dados que permita ao usuário buscar estruturas musicais em obras da
  literatura. }

\newpage

\setcounter{page}{1}
\onehalfspace

\Section{Introdução e objetivos do projeto e do plano de trabalho}

\info{Delimitação do problema trabalhado e as conexões entre o plano
  de trabalho do bolsista e o projeto do orientador. Objetivos e
  justificativa do plano em termos de relevância para a pesquisa
  cientifica e do estado da arte.}

A codificação musical é a representação (gráfica ou não) de elementos
sonoros. Diversas formas de codificação musical têm sido usadas desde
os primórdios da transcrição de sons. Um exemplo simples é a
utilização letras para representar as notas (C para dó, D para ré). A
maioria das pesquisas referentes a este assunto está relacionada a
aplicações para computador \cite[p.~3--5]{selfridge-field97:beyond},
em virtude das possibilidades de processamento oferecidas. Apesar de
tantos trabalhos nesta área, a codificação de estruturas musicais em
um alto nível de abstração, como, por exemplo, retornar ao usuário do
sistema estruturas musicais tais como respostas em fugas ou acordes em
corais, tem ocupado um lugar ainda não preenchido na área de
computação musical

A pesquisa do orientador tem como objetivo desenvolver uma definição
sintática para fragmentos musicais que permita a descrição musical em
níveis mais altos de abstração além da nota/duração, incluindo
estruturas formais e elementos idiosincráticos (como a noção de
sujeito e resposta em fugas, grupos de temas na forma sonata, ou
motivos em certos tipos de composições). É também objetivo da pesquisa
desenvolver o protótipo de um programa de computador para testar e
auxiliar o processo de desenvolvimento dessa definição sintática. Para
este fim são necessárias soluções para definir sintaticamente diversos
tipos de estruturas musicais, bem como exemplos musicais codificados
de acordo com estas soluções.

Esta necessidade justifica o plano de trabalho do bolsista, que tem
como objetivo analisar problemas de definição sintática de estruturas
de alto nível de abstração de um grupo de obras; auxiliar na
elaboração da definição sintática destas estruturas, e codificar
exemplos de estruturas musicais para teste do sistema. É também
objetivo deste plano a divulgação do trabalho em eventos e/ou
publicações científicas.

\Section{Estratégia metodológica}
\info{Descrição da maneira como foram desenvolvidas as atividades para se
chegar aos objetivos propostos. Indicar o material e métodos que foram
usados.}

Para atingir os objetivos do plano de trabalho foi necessário um
treinamento nas ferramentas utilizadas pela pesquisa do orientador.
Este treinamento consistiu em aulas expositivas do orientador, leitura
da documentação das ferramentas, leitura de perguntas e respostas em
grupos de discussão sobre o assunto na internet e na própria prática
com as ferramentas. Apesar de ter sido mais intenso no início da
pesquisa, esta etapa durou todo o ano. Fizemos a documentação de todas
as atividades executadas e comunicação via \eng {logbook} (diário de
pesquisa) e lista de e-mail do grupo de pesquisa, que mantém o
histórico com todos os e-mails referentes à pesquisa. Usamos
basicamente três ferramentas:
\begin{itemize}
\item Controle de versão: Utilizamos para esse fim o software livre
  \eng {Darcs} <\url{http://darcs.net/}>.
\item Sistemas de tipografia de música e de texto: Usamos
  \eng{Lilypond} para música <\url{http://www.lilypond.org}> e \LaTeX
  para texto <\url{http://www.latex-project.org/}>.
\item Editor de texto útil para os sistemas de tipografia e
  codificação LISP: Usamos o \eng{GNU Emacs}.
  <\url{http://www.gnu.org/software/emacs/}>.
\end{itemize}

Após o treinamento e a introdução aos problemas gerais da pesquisa
passamos à etapa de análise dos problemas de codificação musical.
Esta etapa consistiu na análise das estruturas musicais de alto nível
de abstração presentes no conjunto de obras indicado no plano de
trabalho e da listagem das estruturas consideradas de maior relevância
musicológica. Estas tarefas foram feitas pelo bolsista e discutidas
com o orientador. Exemplos destas estruturas são: melodia em grande
escala (c.f. figura \ref{fig:gescala}), contornos melódicos motívicos,
notas auxiliares à harmonia (notas de passagem, bordaduras, retardos,
etc.), espaçamento de acordes e vozes ``estacionárias'' (c.f. figura
\ref{fig:estacionaria}).

\begin{figure}
\begin{minipage}[t]{6.5cm}
 \centering
  \includegraphics{grandeescala}
 \caption{Melodia em grande escala}
 \label{fig:gescala}
\end{minipage}
\hfill
\begin{minipage}[t]{6.5cm}
 \centering
 \includegraphics{estacionaria}
\caption{Voz ``estacionária''}
\label{fig:estacionaria}
\end{minipage}
\hfill
\end{figure}

Seguindo a metodologia \eng{bottom-up} definida no projeto do
orientador, resolvemos não nos concentrar na listagem de todos os
tipos de estruturas musicais para só depois codificá-las. Decidimos
iniciar a criação de sintaxes para algumas poucas estruturas e
codificar alguns exemplos da literatura para testar a própria
sintaxe.

No decorrer da pesquisa, o orientador, junto com o grupo de pesquisa,
escolheu algumas estruturas para o trabalho de definição sintática:
vozes, harmonia e forma de um grupo de peças coral harmonizadas por
J.S.Bach; a orquestração de acordes escritos para orquestra; e
sujeitos, respostas e motivos de um grupo de fugas do Cravo Bem
Temperado, de J.S.Bach. A metodologia empregada nesta etapa também
consistiu em análise das estruturas musicais, sugestões de definições
sintáticas e discussões a respeito destas sugestões.

A metodologia flexível escolhida pelo orientador se justificou no
momento em que apareceu um problema de codificação de solução
complexa, problema este que levou a um breve afastamento das tarefas
previamente planejadas. Tal problema exigiu uma revisão bibliográfica
em \eng{Music Information Retrieval --- MIR} para aprofundamento do
problema específico.

\Section{Atividades executadas no período}
\info{Relação itemizada das atividades executadas, em ordem seqüencial e
temporal, de acordo com os objetivos traçados no plano e dentro do
período de execução do plano.}
\label{sec:atividades}

\begin{itemize}
\item Treinamento com ferramentas. 15.08 a 04.09.2005\footnote{Esta
    tarefa foi mais intensa durante o período citado, mas durou
    durante todo o período da pesquisa.}
\item Introdução aos problemas gerais da pesquisa. 05 a 22.09.2005
\item Seleção de estruturas musicais relevantes para definir
  sintaticamente. 09 a 23.09.2005
\item Definição sintática das vozes e da forma de três corais
  harmonizados por J.S.Bach: Discussão e execução. 26.09 a 07.11.2005
\item Discussões a respeito da seleção vertical de estruturas
  musicais. 17.10.2005 a 10.02.2006
\item Revisão bibliográfica e alimentação da base de dados
  bibliográfica do grupo Genos. (52 artigos MIR). Nov. e Dez./2005
\item Preparação e apresentação de painel no Seminário Estudantil de
  Pesquisa - PIBIC. Nov./2005
\item Definição sintática de acordes de orquestra. Discussão e
  execução. Jan./2006
\item Migração para o sistema operacional Linux e conjunto de
  softwares livres. 18.01 a 13.02.2006.
\item Preparação do relatório parcial PIBIC. 05 a 13.02.2006
\item Preparação do artigo para o VI Workshop de Software
  Livre. Jan. e Fev./2006
\item Seleção de tópico individual para apresentação no Seminário
  Estudantil de Pesquisa - PIBIC 2006. Jan./2006
\item Análise formal e redução analítica das fugas n.1 e 2 do livro 1
  do Cravo Bem Temperado, de J.S.Bach. 15.03 a 24.04.2006
\item Análise harmônica da Sonata p/ piano Op.27 de L.V.Beethoven,
  discussão a respeito de estruturas mais relevantes de uma sonata, e
  proposta de definição sintática da harmonia da Sonata p/ piano Op.27
  de L.V.Beethoven.  31.03 a 13.04.2006
\item Discussão a respeito de adequação do artigo escrito para o
  WSL2006 para a ANPPOM 2006. 23.05.2006
\item Proposta de definição sintática para estruturas relevantes de
  uma fuga. 02.06.2006
\end{itemize}

\Section{Resultados e Discussão}
\info{Relação dos resultados ou produtos obtidos durante a execução da
pesquisa, indicando os avanços no conhecimento disponível obtidos com
a execução da pesquisa.}

Os principais resultados desta pesquisa foram:

\begin{itemize}
\item Definição sintática das vozes, harmonia e forma de um grupo de
  peças coral harmonizadas por Johann Sebastian Bach. (c.f. figuras
  \ref{fig:sintform} e \ref{fig:sintharm})

  Esta definição permite uma série de possibilidades de processamento
  dos dados codificados.  Como a sua arquitetura identifica os
  parâmetros das estruturas musicais, é possível reconhecer elementos
  de tais estruturas. Por exemplo, na sintaxe da harmonia, os acordes
  têm sua tipologia, inversão e função harmônica codificados. É
  possível com isso encontrar em uma peça codificada, acordes de um
  tipo específico, como um diminuto, sem que seja necessário analisar
  a partitura ou identificar auditivamente. Pode-se ainda facilmente
  reunir todos os diminutos e compará-los para observar
  características comuns ou diferenças. Cruzando dados codificados de
  harmonia e forma pode-se verificar, por exemplo, todas as soluções
  harmônicas encontradas pelo compositor para as cadências de engano.
  Por estes motivos esta definição se mostra bastante útil para
  análise musical.

  Nas figuras \ref{fig:sintform} e \ref{fig:sintharm} percebe-se que
  as sintaxes criadas contemplam a hierarquia musical entre períodos e
  frases, tonalidades e regiões cêntricas. Isto foi possível graças ao
  conceito de evento, elemento que pode ser hierarquicamente
  organizado.  Pode-se notar na figura \ref{fig:sintform}, por
  exemplo, que o evento coral-5 contém o evento período 1. Este contém
  o evento frase 1. Já este último contém os eventos soprano,
  contralto, tenor e baixo. Cada evento pode conter parâmetros, que
  são indicados pelo símbolo ``:''. (e.g. :notes).

\begin{figure}
  \centering
  \footnotesize
\begin{verbatim}
(event 'coral-5
  :key '(c major)
  :time 4/4
  (event 'periodo1
    (event 'frase1
      :partial 4
      (event 'soprano
        :relative 5
        :notes "c c8 b a b c d e4 d d c")
      (event 'contralto
        :relative 4
        :notes "notas")
      (event 'tenor
        :relative 4
        :notes "notas")
      (event 'baixo
        :relative 3
        :notes "notas"))[...]))
\end{verbatim}
  \caption{Sintaxe para vozes e forma dos corais}
  \label{fig:sintform}
\end{figure}

\begin{figure}
  \centering
  \footnotesize
\begin{verbatim}
(event 'coral-4
  (event 'periodo 1
    (frase 1
      :partial 4
      (harmony
       (center c(vi minor)(i 1)(iv)(i 1)[...])))
    (frase 2
      :partial 4
      (harmony
       (center g(vii dim 1)(i))
       (center c(ii minor)(vi8 minor)[...])))))
\end{verbatim}
  \caption{Sintaxe para harmonia dos corais}
  \label{fig:sintharm}
\end{figure}

Os elementos ``forma'', ``harmonia'' e ``vozes'', no entanto, não
estão relacionados entre si de forma hierárquica. Eles são, ao mesmo
tempo, dependentes e independentes entre si. (c.f. figura
\ref{fig:hcoral}). Este tipo de relação pode ser entendida pelo
seguinte exemplo: as vozes não estão sub-relacionadas à harmonia ou à
forma, porém a forma e a harmonia são o resultado do comportamento
das vozes. Podemos pensar também que as vozes se comportam de acordo
com uma determinada forma. Um outro exemplo é a harmonia, que é
pensada a partir do baixo mas não depende necessariamente dele. Basta
saber o tipo de acorde e a inversão para codificá-la. 

\begin{figure}
  \centering
  \includegraphics[scale=.8,angle=90]{hierarquias}
  \caption{Hierarquias do coral}
  \label{fig:hcoral}
\end{figure}

\item Definição sintática para sujeitos, respostas e motivos de um
  grupo de fugas do Cravo Bem Temperado, de J.S.Bach.

Este definição tem mesmo princípio, características e funcionalidades
da definição das vozes, harmonia e forma dos corais.

\item Conhecimento superficial de sintaxes de programação: Lilypond e
  LISP.

  Com a prática no uso destas sintaxes de programação, o bolsista
  adquiriu conhecimentos básicos para criar as codificações propostas
  no plano de trabalho, bem como para editar a maioria das peças da
  literatura musical.

\item Treinamento com ferramentas.

  Este foi um dos mais importantes resultados desta pesquisa. As
  ferramentas nela utilizadas têm lugar em qualquer investigação
  científica, seja da área de Música ou não. A documentação das
  atividades através de \eng{logbook} e lista de e-mails tornou fácil,
  rápida e precisa toda busca por informação referente aos passos do
  trabalho. Exemplo disso é a precisão de data das atividades
  realizadas listadas na seção \ref{sec:atividades}. O uso do controle
  de versão permitiu perfeitamente o trabalho simultâneo e
  não-presencial em grupo. Essa ferramenta permite que se edite
  arquivos sem que se perca nenhuma informação, já que a cada edição
  ela registra apenas o conteúdo modificado, o usuário que modificou,
  nome do arquivo, data e hora, e envia a tais dados a um repositório
  central que é acessado por todos. Cada um baixa apenas o conteúdo
  modificado e atualiza seu repositório individual. Exemplo do uso
  desta ferramenta é o artigo escrito para a ANPPOM 2006 (c.f. seção
  \ref{sec:artigos}), que foi feito em co-autoria por todo o grupo de
  pesquisa. Os sistemas de tipografia musical e de texto também têm
  funcionalidades muito interessantes. Com o sistema \LaTeX, por
  exemplo, é possível escrever um texto sem a preocupação com a
  formatação final, que pode seguir padrão ABNT, Turabian, PIBIC ou
  qualquer outro. Isto é possível porque esta ferramenta trabalha com
  marcações e pacotes de estilos (como HTML e CSS).  Para mudança de
  estilo basta selecionar o pacote desejado. As funcionalidades destas
  ferramentas se mostraram tão sedutoras que levaram o bolsista a
  mudar completamente seu conceito de uso de computador.


\item Linux e software livre

  A utilização do Linux e de outros softwares (inclusive todas as
  citadas ferramentas) que são reconhecidos como livres foi outro
  resultado importante desta pesquisa. Todos os softwares utilizados
  têm licença livre, algo que, além de ter custo praticamente
  inexistente, se mostra como uma forte tendência. É possível, com
  essa escolha, racionalizar a verba utilizada pela pesquisa.

\item \eng{Music Information Retrieval --- MIR}

Foi possível ao bolsista, a partir da revisão bibliográfica executada,
conhecer um pouco do objeto de pesquisa e alguns dos problemas
relacionados ao \eng{MIR}.

\item Resultados parciais: definição da harmonia da sonata.

Pode-se considerar como resultado parcial a definição sintática usada
para a harmonia de uma sonata, pois, apesar desta definição ainda
requerer maiores discussões, já construímos um esboço do qual pode-se
partir para a definição final.

\end{itemize}

\Section{Considerações finais}
\info{Expor de modo sucinto a contribuição do seu projeto ao projeto de
pesquisa do orientador e ao conhecimento científico da sua área,
apresentando as implicações para futuros trabalhos que podem ser
desenvolvidos.}

O trabalho desenvolvido pelo bolsista contribui com o projeto do
orientador através da criação de soluções para a definição sintática
de estruturas musicais relevantes e de alto grau de abstração,
referentes a uma parte das obras propostas pelo plano de trabalho
inicial. Contribui também através da codificação de peças que têm sido
úteis para os testes do protótipo do software em desenvolvimento.

Futuramente, com os produtos desta pesquisa, será possível a criação
de um software livre que retorne ao seu usuário as estruturas musicais
de alto grau de abstração que trabalhamos até então e até mesmo outras
estruturas com as quais possamos trabalhar posteriormente. Pode-se
também criar um banco de dados que contemple obras significativas da
literatura musical, codificadas a partir das definições desenvolvidas.

\Section{Dificuldades e soluções}
\info{Expor as dificuldades enfrentadas no desenvolvimento do plano e
  as estratégias utilizadas para sua resolução.}

\paragraph{Estruturas musicais verticais}
\label{sec:estr-music-vert}

O principal e mais complexo problema desta pesquisa foi a definição
sintática de estruturas musicais verticais. Apesar destas estruturas
serem entendidas verticalmente, seu código é escrito de forma
horizontal.  (c.f. figura \ref{fig:codselvert}). Para selecionar um
evento que tenha sua estrutura vertical é necessário combinar a
posição exata de suas notas em cada voz codificada.  A figura
\ref{fig:selvert} traz um pequeno trecho de textura a quatro vozes com
um acorde destacado (a estrutura vertical), e a figura
\ref{fig:codselvert}, a codificação deste trecho com um destaque para
a referida seleção.

\begin{figure}
\begin{minipage}[t]{6.5cm}
  \centering
 \includegraphics{selecao-vertical-notas}
 \caption{Seleção vertical em notação}
 \label{fig:selvert}
\end{minipage}
\hfill
\begin{minipage}[t]{6.5cm}
  \centering
 \includegraphics[scale=.9]{selecao-vertical-codigo}
 \caption{Seleção vertical em código}
 \label{fig:codselvert}
\end{minipage}
\hfill
\end{figure}

Discutimos formas de definir sintaticamente uma estrutura vertical e
percebemos, a priori, que são possíveis duas formas de codificação.  É
possível utilizar a posição das notas da estrutura na codificação de
cada voz. Isto significa observar a exata posição que tais notas
ocupam no código da música ou do fragmento musical. Por exemplo, a
estrutura codificada na figura \ref{fig:codselvert} seria localizada
com uma sintaxe como: (L3, E4; L6, E3; L9; E3 e L12, E3), onde L =
linha e E = entrada. Esta sintaxe tem como aspecto negativo a
dificuldade de alteração no código da música. Por exemplo, na figura
\ref{fig:selvert}, se for necessário substituir as duas notas
\textit{mi} de valor igual a semínima da parte do contralto por quatro
notas de valor igual a colcheia, teremos esta codificação:

\begin{verbatim}
e8 e e e f4 f
\end{verbatim}

Com isso, a sintaxe da estrutura musical muda de L6, E3 para L6, E5.

Uma outra forma de definir sintaticamente uma estrutura vertical é
utilizar uma espécie de endereço do evento, que seria, no caso da
figura \ref{fig:selvert}, algo como:

\begin{verbatim}
soprano {1 4; 1 5}
contralto {1 3; 1 4}
tenor {1 3; 1 4}
baixo {1 3; 1 4}
\end{verbatim}

Neste caso o conteúdo entre chaves significa, respectivamente,
\{início do evento (número do compasso) (tempo); fim do evento (número
do compasso) (tempo)\}. Esta sintaxe tem duas desvantagens: a
prolixidade, já que deverá conter o exato ponto inicial e final da
estrutura vertical em cada voz envolvida; e a dificuldade de
codificação de ritmos com quiálteras complexas.

Este problema exigiu aprofundamento de estudos na bibliografia
disponível, porém, mesmo com a revisão bibliográfica não foi possível
encontrar uma solução convincente. Isto levou o orientador a elaborar
um novo plano de trabalho para o período 2006-2007 para que o bolsista
possa aprofundar os estudos acerca do tema para definitivamente
solucionar o problema.

\paragraph{Codificação de ritmos com quiálteras complexas}
\label{sec:codif-de-ritm}

Esta foi outra dificuldade encontrada durante a pesquisa. A
codificação de ritmos binários simples foi resolvida com um padrão
numérico em que uma semibreve é equivalente ao número 1, uma mínima,
ao número 2, uma semínima ao número 4, e assim por diante. Contudo,
para um ritmo com quiálteras como o da figura \ref{fig:ritmocomp} não
é possível utilizar esta idéia. Estamos utilizando a definição criada
por Jamary Oliveira, que é representada na equação \ref{eq:1}, de
forma temporária, já que esta ainda se mostra bastante complexa para
os nossos propósitos. Este tópico ainda precisa ser aprofundado.

\begin{figure}
  \centering
  \includegraphics{ritmo}
  \caption{Ritmo de quiálteras complexas}
  \label{fig:ritmocomp}
\end{figure}

\begin{equation}
  \label{eq:1}
  \frac{3}{8},\frac{1}{8}+,\frac{2}{3}\left[\frac{1}{8},\frac{1}{8}+,\frac{4}{5}\left[\frac{1}{16},\frac{3}{16},\frac{1}{16}\right]\right]
\end{equation}

\paragraph{Configuração do computador}
\label{sec:conf-do-comp}

Outro problema que houve nesta pesquisa foi a configuração do
computador do bolsista para o uso das ferramentas da pesquisa. Esta
configuração consistiu na instalação das ferramentas no sistema
operacional Windows, instalação e configuração do sistema operacional
Linux (distribuição Ubuntu <\url{http://www.ubuntu.com/}>) e
instalação e configuração das ferramentas no sistema Linux. A
principal razão desta dificuldade foi o desconhecimento, por parte do
bolsista, das ferramentas utilizadas. O problema foi resolvido através
de aulas e suporte do orientador, estudo da documentação do sistema,
leitura de perguntas e respostas nas listas de discussão sobre o tema
na internet, e, principalmente, com a utilização diária do sistema.

%%%%%%%%% bibliografia %%%%%%%%%%%%%%%%
\renewcommand{\refname}{Referências bibliográficas  (máximo 15)}
\info{Relação itemizada das referências que subsidiam a proposta de
pesquisa, colocando as mais importantes.}

\nocite{ismir2003,ismir2002,ismir2001,ismir2000,kroger05:codificacao,doclilypond,docdarcs,stallman02:emacs}
\bibliographystyle{plain}
\bibliography{bibliography}

\Section{Participação em reuniões científicas e publicações}
\info{Relacionar as reuniões científicas e os títulos dos trabalhos
apresentados pelo estudante durante a vigência da bolsa. Incluir
títulos de publicações que resultaram ou se beneficiaram de seu
trabalho.}

\begin{itemize}
\item Apresentação de poster no VI Seminário de Pesquisa e
  Pós-Graduação, realizado entre os dias 9 e 12 de novembro de 2005:
  \textit{Desenvolvendo uma codificação para definir estruturas
    musicais}.

\item Submissão de artigo para o XVI congresso da Associação Nacional
  de Pesquisa e Pós-Graduação em Música --- ANPPOM, que se realizará
  em Brasília, entre 27.08 e 01.09.2006 (esperando aprovação):
  \textit{O processo de desenvolvimento de uma codificação para
    definir estruturas musicais}
\end{itemize}

\Section{Anexos}
\info{Anexar os resumos ou trabalhos que foram apresentados pelo bolsista
durante a vigência da bolsa.}
\label{sec:artigos}

\textbf{O processo de desenvolvimento de uma codificação para definir
estruturas musicais}

Resumo:

O trabalho de codificação de estruturas musicais com um alto grau de
abstração ocupa um espaço ainda não preenchido na área de computação
musical. Esse artigo descreve a pesquisa onde linguagens de domínio
específico estão sendo criadas em Lisp para testar possibilidades de
codificações segundo um modelo bottom-up. Ou seja, diferentes aspectos
de uma codificação são testados em um protótipo e sua utilidade é
validada ou descartada. Até o momento alguns sub-problemas foram
identificados com a ajuda dessa abordagem, como a utilização de
eventos, codificação das durações, e codificação vertical. O uso de
ferramentas específicas de software livre colabora com a organização e
dinamismo dessa pesquisa e todo o código gerado é licenciado sob a
Licença Pública Geral (GPL).

\textbf{Desenvolvendo uma codificação para definir estruturas musicais}

\includegraphics[scale=.15]{poster-2005}

\end{document}
