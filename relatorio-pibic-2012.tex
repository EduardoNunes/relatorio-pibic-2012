\documentclass[a4paper,12pt]{article}

\usepackage{ucs}
\usepackage[utf8]{inputenc}
\usepackage[brazil]{babel}
\usepackage[T1]{fontenc}
\usepackage{graphicx}
\usepackage{url}

\usepackage[dvips]{hyperref}

\author{Eduardo Lago Nunes}
\title{Relatório PIBIC}
\date{07/04/12}

\begin{document}

\maketitle

\section{Introdução}
\label{sec:introducao}
% importância de contornos para música, da teoria de contornos, a inconsistência...


% nas seções seguintes descrever as coisas
\section{Materiais e métodos}
\label{sec:materiais}
% materiais: artigos, MusiContour (dizer que faz parte da pesquisa), planilha
% métodos: como fez cada coisa

MusiContour\footnote{Disponível em \url{http://genosmus.com/MusiContour}.}

\section{Resultados}
\label{sec:resultados}
% planilha com mapeamento de operações e o que funciona
% aprendizado


\section{Discussão}
\label{sec:discussao}
% comentar o que foi dito nas outras seções. ir além da descrição

\end{document}
