%% Local IspellDict: brasileiro

\documentclass[11pt]{article}
\usepackage{relatorio-pibic}
\usepackage[utf8x]{inputenc}
\usepackage{cite}

% usar para palavras estrangeiras: \eng{english word}
\newcommand{\eng}[1]{\textit{#1}}

\begin{document}

\graphicspath{{figs/}}

\cabecalho{Relatório Final}

\dadosRelatorioFinal
{Levantamento de operações de teorias de contornos em análises de
  obras musicais}
{Verificando inconsistências nas teorias de contornos musicais através
  de ferramentas computacionais }
{Eduardo Lago Nunes}
{Marcos da Silva Sampaio}
{Genos}
{Inserir 3 palavras chave}
{JANEIRO A JULHO DE 2012}

\resumo{Inserir texto do resumo}

\newpage

\setcounter{page}{1}
\onehalfspace

% Marcos: orientações gerais:

% Inicialmente não se preocupe tanto com o texto, mas com os dados
% para usar no texto. Depois que tiver a lista dos materiais, das
% etapas cumpridas, dos resultados, das dificuldades, você pode
% refletir sobre cada um e escrever anotações sobre eles. Em seguida
% você vai transformando essas anotações no seu texto

% Foque primeiro nos materiais e métodos. O que você fez, como você
% fez. Arrume esta seção antes das outras

% O seu leitor pode não entender nada de contornos. Com o seu texto
% ela precisa entender que havia um problema para ser
% resolvido---simplificar os testes e verificação de operações de
% contornos---e você resolveu com a planilha

% Não use vírgula entre sujeito e verbo

\Section{Introdução}
\label{sec:introducao}
\info{Delimitação do problema trabalhado e as conexões entre o plano
  de trabalho do bolsista e o projeto do orientador. Objetivos e
  justificativa do plano em termos de relevância para a pesquisa
  cientifica e do estado da arte.}
% importância de contornos para música, da teoria de contornos, a
% inconsistência...

% Marcos: O que são contornos? Você sempre pode parafrasear alguém e
% citar.
Contornos são importantes pelo fato de fazer ser mais perceptível uma
certa melodia ou ritmo que esteja sendo executado, até mesmo para
pessoas com ouvido destreinado.

% Marcos: Qual a sua ideia neste parágrafo? Qual ferramenta? De quem é
% a teoria?

% Marcos: Remover vírgula entre sujeito e verbo
A teoria de contornos visa estudar esta ferramenta com intuito de
desenvolver técnicas para análises do estudo dos padrões e
comportamento destes contornos.
% Marcos: quem vai desenvolver? Que pesquisas são essas? É o projeto
% do PIBIC? É outra coisa?
A partir destes estudos, desenvolver uma solução para uma
inconsistência encontrada durante as pesquisas do orientador.

% nas seções seguintes descrever as coisas
\Section{Materiais e métodos}
\label{sec:materiais}
\info{Descrição da maneira como foram desenvolvidas as atividades para
  se chegar aos objetivos propostos. Indicar o material e métodos que
  foram usados.}

% Marcos: pense no objetivo geral do seu plano de trabalho para ter
% mais facilidade de escrever este trecho: "Levantamento de operações
% de teorias de contornos em análises de obras musicais". A planilha é
% um recurso que você usou para fazer este levantamento de
% operações.

% Marcos: Não precisa dizer que não havia tempo hábil. Basta dizer que
% você realizou o trabalho usando estes materiais.

Para o desenvolvimento da planilha de operações houve a necessidade de
utilizarmos diversos materias, pois, havia pouco tempo hábil para assimilar
treinamento e trabalho.

% Marcos: Pensei melhor e acho melhor você fazer uma lista com uma
% linha de explicação para cada um:
% mendeley \footnote{link}. Repositório colaborativo de artigos da
% pesquisa
% MusiContour \footnote{link}. Software de processamento de operações
% de contornos.....

Materiais:
Planilha de operações.
Mendeley\footnote{Disponível em
  \url{http://www.mendeley.com/}.}, Artigos\footnote{Disponível em
  \url{http://www.mendeley.com/groups/1138861/_/papers/}.},
  MusiContour\footnote{Disponível em \url{http://genosmus.com/MusiContour}.},
  Python\footnote{Disponível em  \url{http://www.python.org/getit/}.},
  Linux\footnote{Disponível em
  \url{http://www.ubuntu.com/download/desktop}.},
  Git\footnote{Disponível em
  \url{http://git-scm.com/downloads}.}, Github\footnote{Disponível em
  \url{https://github.com/}.}, Kile e Latex\footnote{Disponível em
  \url{http://kile.sourceforge.net/}.}.

Utilidades dos materiais: Artigos sobre teoria de contornos, material da pesquisa
que foi analizado para a extração dos contornos. Mendeley, local onde o grupo de pesquisa
armazena os arquivos que foram estudados. MusiContour, software desenvolvido pelo orientador
que faz os cáculos das operações de contornos. Python, editor e linguagem de programação em
que foi desenvolvido o software MusiContour. Linux sistema operacional utilizado durante a pesqisa.
Git, sistema de controle de versão distribuído com ênfase em velocidade. Github, Serviço de Web
Hosting Compartilhado para projetos que usam o controle de versionamento Git, e também onde organizamos
as tarefas diárias. Kile, editor de código TeX/LateX. Latex, conjunto de macros para o processador de
textos. Por fim, a planilha de operações que é o principal material e também o resultado do trabalho.

% Marcos. Remova o 'foi necessário' e apenas diga que para realizar o trabalho
% você cumpriu as etapas listadas
Foi necessário também uma metodologia de trabalho/treinamento para que fosse possível obter
o resultado desejado tanto para o aprendizado das ferramentas a utilizar, a absorção de conhecimento
e o objetivo do trabalho que é a planilha de operações.

% Marcos. Coloque uma ou duas linhas em cada item para explicar o que é.
\begin{enumerate}
\item Treinamento.
\item Leitura e elaboração de resenhas dos artigos.
% Marcos. Alimentação da planilha de operações fica melhor
\item Inserção das operações na planilha de operações.
% Marcos. Junte instalação de ubuntu com treinamento em python e musicontour e
% chame de instalação e treinamento em ferramentas.
\item Instalação do Ubuntu.
\item Treinamento para python e MusiContour.
\item Revisão das operações contidas na planilha
% Marcos. Remova a elaboração do relatório. Em um projeto acho
% necessário ter esse item, mas no próprio relatório não.
\item Elaboração do relatório.
\end{enumerate}

% Marcos. Como falei, coloque essas informações direto na lista
% Marcos. Você pode dizer que o treinamento consistiu na leitura da
% literatura sobre contornos, produção de resenhas e sessões de
% dúvidas com o orientador. Mova as dificuldades encontradas para a
% parte de discussão
Primeiramente houve o treinamento explicando o que é contorno, qual a proposta do trabalho e
do que se tratava a pesquisa. Em seguida foram apresentados os artigos e como eram dispostas
as operações de contornos. Neste passo foram encontradas dificuldades como a língua inglesa (todos
os artigos estão escritos em inglês), e o entendimento de algumas operações. Estas dificuldades
foram esclarecidas com a leitura dos textos, eplicações em reuniões e elaboração de resenhas.

% Marcos. Neste local apenas informe no que consiste a planilha. Tente
% não passar de duas linhas. Fale sobre a necessidade de ser revisada
% e formatada na parte de discussão. Siga esse princípio para as
% outras partes

% Marcos. Você usa a expressão contexto histórico-musical, mas não
% define o que isso é. Em casos assim você primeiro precisa definir a
% expressão para depois usar. De qualquer forma eu não concordo com
% essa expressão. Basta dizer que você mapeou as operações de acordo
% com a referência de origem.
Havia uma planilha com algumas operações já inclusas, porém ela precisava ser revisada e estilizada.
Todos os artigos foram analizados e todas as operações relacionadas ao contexto historico-musical
encontradas nos artigos foram adicionados à planilha de operações. Concluído esta etapa, bastava analizar
cada operação inclusa na planilha separadamente. Como haviam muitas operações e algumas necessitavam de
cáculos extensos e demorados, foi utilizado o software MusiContour. Para que este software pudesse ser
utilizado era necessário um treinamento para Python, para o MusiContour e também a instalação do
sistema operacional Ubuntu. O Ubuntu foi utilizado para agilizar a instalação e utilização dos proramas
necessários. Como eu utilizava Windows estávamos perdendo tempo tentando descobrir comandos de instalações
e traduções de linguagem do Ubuntu para Windows.
Finalmente para elaborar o relatório utilizei o software kile, que permite uma facilidade no trabalho
de ediçã, um aprendizado adicional sobre programação, e um aprofundamento nos softwares citados em materiais.

\Section{Resultados}
\label{sec:resultados}
\info{Relação dos resultados ou produtos obtidos durante a execução da
  pesquisa, indicando os avanços no conhecimento disponível obtidos
  com a execução da pesquisa.}

% planilha com mapeamento de operações e o que funciona
% aprendizado

% Marcos: coloque esse comentário na parte de discussão. Aqui você
% apenas diz quais os resultados.
A pesquisa foi de grande valia para a edificação do meu conhecimento e também para o trabalho do orientador.
Além do resultado principal, e do conhecimento adiquirido, houveram resultados secundarios também de grande
importância.

% Marcos. Não diga aqui que foi cumprido no prazo. Isso dilui o
% objetivo mais importante, de entender o resultado.
A planilha de operações de contornos foi o principal objetivo do tabalho e que foi cumprido no prazo. A planilha
de operações é um catálogo onde estão mapeadas todas as operações, que fazem parte do contexto histórico-musical,
contidas nos artigos estudados.

% Marcos. Antes desta lista diga que este trabalho também teve
% resultados secundários. Coloque todos menos a planilha
\begin{enumerate}
\item Planilha de operações.
% Marcos. aprimoramento da leitura em língua inglesa
\item Aprimoramento da leitura inglesa.
% Marcos. padronize a forma de falar. Aprofundamento do conhecimento
% sobre contornos.
\item Conhecimento aprofundado sobre contornos.
\item Aprendizado sobre Ubuntu, Python, Kile, MusiContour.
% Marcos. Aprimoramento da escrita de textos acadêmicos
\item Escrita e consisão de resenhas.
% Marcos. Aprimoramento de organização e técnicas de estudo
\item Organização e técnicas de estudo.
\end{enumerate}


\Section{Discussão}
\label{sec:discussao}
\info{Expor de modo sucinto a contribuição do seu projeto ao projeto
  de pesquisa do orientador e ao conhecimento científico da sua área,
  apresentando as implicações para futuros trabalhos que podem ser
  desenvolvidos.}

% comentar o que foi dito nas outras seções. ir além da descrição
% EDUARDO não mexi nessa parte ainda.
A pesquisa me possibilitou um contato direto com diversos artigos que abordavam
o tema contornos. Este tipo de contato que não se encontra no curso de graduação,
me trouxe um conhecimento aprofundado sobre a teoria de relações de contornos muscais.
Me possiblitou também, um aprimoramento na leitura da lígua inglesa, escrita e conscisão
de resenhas, e um conhecimento sobre diversas ferramentas computacionais.

Esta pesquisa tem sua importância em reduzir significativamente o
tempo de trabalho do orientador. Como todas as operações estarão mapeadas
na planilha, basta olhar onde se localiza e como está descrita a operação
desejada. O catálogo indica o artigo, a página, o nome da obra, o compositor,
se está em gráfico ou texto e em que parágrafo ou figura. Assim, basta escolhar
uma operação e ir diretamente no artigo, pagina e parágrafo indicado pelo
catálogo.
% Marcos: O parágrafo está solto. O 'ainda assim' pressupõe que a
% ideia deste parágrafo é adversativa a algum outro, mas não há outro
% parágrafo.

% Marcos: Prefira sempre a ordem direta das frases e evite a voz
% passiva. Diga que você encontrou incoerências após testar as
% operações. Substitua o "devidas providências" por algo como
% "posterior avaliação"
Ainda assim, após a análise detalhada de cada operação em particular,
foram encontradas incoerências, estas serão estudadas para que sejam
tomadas as devidas providências.

% Marcos: Não coloque a bibliografia por extenso aqui. Eu irei
% formatar a bibliografia automaticamente depois. Se quiser
% acrescentar algum livro, apenas coloque o nome do autor e o ano como
% um comentário, dessa forma:

% Morris 1993

%%%%%%%%% bibliografia %%%%%%%%%%%%%%%%
% insere bibliografia

\renewcommand{\refname}{Referências bibliográficas (máximo 15)}
\info{Relação itemizada das referências que subsidiam a proposta de
  pesquisa, colocando as mais importantes.}

\nocite{
  Friedmann1985,
  Friedmann1987,
  Morris1987,
  Marvin1988,
  Polansky1992,
  Morris1993,
  Clifford1995,
  Quinn1997,
  Beard2003,
  Sampaio2008,
  Schultz2008,
  Schultz2009,
  Bor2009
}

\bibliographystyle{plain}
\bibliography{bibliography}

\Section{Participação em reuniões científicas e publicações}
\info{Relacionar as reuniões científicas e os títulos dos trabalhos
apresentados pelo estudante durante a vigência da bolsa. Incluir
títulos de publicações que resultaram ou se beneficiaram de seu
trabalho.}

\Section{Anexos}
\info{Anexar os resumos ou trabalhos que foram apresentados pelo bolsista
durante a vigência da bolsa.}

\end{document}
