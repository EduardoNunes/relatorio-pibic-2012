%% Local IspellDict: brasileiro

\documentclass[11pt]{article}
\usepackage{relatorio-pibic}
\usepackage[utf8x]{inputenc}
\usepackage{cite}

% usar para palavras estrangeiras: \eng{english word}
\newcommand{\eng}[1]{\textit{#1}}

\begin{document}

\graphicspath{{figs/}}

\cabecalho{Relatório Final}

\dadosRelatorioFinal
{Levantamento de operações de teorias de contornos em análises de
  obras musicais}
{Verificando inconsistências nas teorias de contornos musicais através
  de ferramentas computacionais }
{Eduardo Lago Nunes}
{Marcos da Silva Sampaio}
{Genos}
{Teoria de Relações de Contornos Musicais, Teoria Musical, Computação Musical}
{JANEIRO A JULHO DE 2012}

\resumo{Inserir texto do resumo}

\newpage

\setcounter{page}{1}
\onehalfspace

% Marcos: orientações gerais:

% Inicialmente não se preocupe tanto com o texto, mas com os dados
% para usar no texto. Depois que tiver a lista dos materiais, das
% etapas cumpridas, dos resultados, das dificuldades, você pode
% refletir sobre cada um e escrever anotações sobre eles. Em seguida
% você vai transformando essas anotações no seu texto

% Foque primeiro nos materiais e métodos. O que você fez, como você
% fez. Arrume esta seção antes das outras

% O seu leitor pode não entender nada de contornos. Com o seu texto
% ele precisa entender que havia um problema para ser
% resolvido---simplificar os testes e verificação de operações de
% contornos---e você resolveu com a planilha
% Ao invés de focar em falar da planilha, fale do mapeamento de
% operações

% Não use vírgula entre sujeito e verbo

\Section{Introdução}
\label{sec:introducao}
\info{Delimitação do problema trabalhado e as conexões entre o plano
  de trabalho do bolsista e o projeto do orientador. Objetivos e
  justificativa do plano em termos de relevância para a pesquisa
  cientifica e do estado da arte.}
% importância de contornos para música, da teoria de contornos, a
% inconsistência...

% Marcos: a ideia geral para esta parte está boa: o que são contornos
% e porque são importantes, teoria de contornos e
% inconsistência. Precisamos apenas desenvolver cada uma dessas partes
% Eduardo: Ok

% Marcos: ainda não dá para entender o que é contorno com esta
% definição. É importante falar o que é contorno graficamente,
% musicalmente para depois dar a definição. Dê uma olhada na minha
% dissertação.
% Eduardo: eu poderia colocar uma citação sua conforme eu fiz? Sua
% explicação tá mais fácil de ser entendida que a de Marvin.
% E também poderia adicionar imagem em alumas partes como essa.
Contorno é o perfil, desenho ou formato de um objeto. Contornos podem ter duas ou
mais dimensões, e podem relacionar altura a comprimento, largura ou tempo. Em música
contornos podem ser abstraídos da altura, densidade, ritmo, timbre, intensidade, etc.
Contornos melódicos, por exemplo, são abstrações do movimento de altura de notas em
função do tempo.
\cite[p. 01]{Sampaio2008}.

% Marcos: Veja a forma de citar um artigo. Você pode olhar essas keys
% lá no final, na parte de bibliografia
% Eduardo: Ok

% Marcos: Quais estudos? Se mencionou tem que citar
% Eduardo: Mas marvin não cita no dele. Eu poderia colocar como está em baixo?
Estudos provam que contornos são importantes pelo fato de fazer ser
mais perceptível uma certa melodia ou ritmo que esteja sendo
executado, até mesmo para pessoas com ouvido destreinado.
\cite[p. 225]{Marvin1987}.

% Marcos: quais pesquisadores? Da mesma forma, se mencionou tem que
% citar
% Eduardo: Se eu colocar como está precisa mencionar algo?
A teoria de contornos visa o estudo do padrão e comportamento destes contornos.

% Marcos: quem vai desenvolver? Que pesquisas são essas? É o projeto
% do PIBIC? É outra coisa?
% Eduardo: Coloquei estas questões integrada ao parágrafo seguinte.

% Marcos: que estudos? que múltiplas teorias são essas? você falou em
% uma teoria só no parágrafo anterior.
% Eduardo: Este comentário aqui poderia ser enquadrado no parágrafo seguinte 
% juntamente com o comentário de cima?

% Marcos: eu encontrei a inconsistência ao programar o
% MusiContour. Veja o projeto que eu submeti ao PIBIC no ano passado
% para ter alguma ideia nessa parte.
% Eduardo: Qual é o projeto?

% Marcos: revise todos os locais onde houver 'planilha de operações' e
% veja se o correto não é 'mapeamento de operações'. Sugiro fazer logo
% isso antes de seguir e fazer um commit separado só para isso. Veja o
% email

% Marcos: pense no objetivo geral do seu plano de trabalho para ter
% mais facilidade de escrever este trecho: "Levantamento de operações
% de teorias de contornos em análises de obras musicais". A planilha é
% um recurso que você usou para fazer este levantamento de
% operações.

% Marcos: acrescente em um parágrafo curto o que o projeto desta
% pesquisa propõe e como seu plano de trabalho se encaixa neste
% objetivo.
% Eduardo: Acrescentei o parágrafo que estava em discussões.

% Marcos: é melhor falar mencionar parte dessa informação na
% introdução. tem um comentário lá
% Eduardo: Coloquei na introdoção.

% Marcos: uma orientação geral, ao invés de pensar que tarefas
% trabalhosas atrasam o projeto, pense que automatizar tarefas
% trabalhosas economizam tempo e energia que podem ser melhor
% aproveitados.
% Eduardo: Ok
Ao criar e testar software MusiCountour, o orientador acabou encontrando uma inconsistência
na teoria dos contornos. Durante os estudos sobre esta inconsistência foi necessário a busca 
por operações de contornos nos artigos. Como existem muitos artigos e estes mesmos possuem 
entre 50 e 300 páginas, automatizar as tarefas trabalhosas economizaria tempo e energia 
que poderiam ser melhor aproveitados. O objetivo do trabalho foi criar esta automatização 
em forma de mapeamento de operações.

% nas seções seguintes descrever as coisas
\Section{Materiais e métodos}
\label{sec:materiais}
\info{Descrição da maneira como foram desenvolvidas as atividades para
  se chegar aos objetivos propostos. Indicar o material e métodos que
  foram usados.}

% Marcos: corrigir digitação. Melhor: '...usamos os seguintes
% materiais:'
% Eduardo: Ok.
Para o desenvolvimento do mapeamento de operações usamos os seguintes materias.

% Marcos: Siga o modelo. Descreva cada item em uma ou duas linhas

\begin{enumerate}
\item Planilha de mapeamento de operações de contornos. Esta planilha
  contém campos [preencher]. Vide mais informações na seção de
  resultados.
\item Mendeley\footnote{Disponível em
    \url{http://www.mendeley.com/}}. Repositório colaborativo dos
  textos da pesquisa. Este programa foi necessário para o
  compartilhamento da bibliografia.
\item MusiContour\footnote{Disponível em
    \url{http://genosmus.com/MusiContour}.}. Software de processamento
  de operações de contornos desenvolvido pelo orientador. Esta
  ferramenta foi necessária para realizar testes de operações de
  contornos.
\end{enumerate}

Material da pesquisa que foi analizado para a extração dos contornos.
Planilha de operações. É o principal material e também o
objetivo do trabalho.
MusiContour\footnote{Disponível em
  \url{http://genosmus.com/MusiContour}.}. Software desenvolvido pelo
orientador que faz os cáculos das operações de contornos.
Python\footnote{Disponível em
  \url{http://www.python.org/getit/}.}. Editor e linguagem de
programação em que foi desenvolvido o software MusiContour.
Linux\footnote{Disponível em
  \url{http://www.ubuntu.com/download/desktop}.}. Linux sistema
operacional utilizado durante a pesquisa.  Git\footnote{Disponível em
  \url{http://git-scm.com/downloads}.}. Sistema de controle de versão
distribuído com ênfase em velocidade.  Github\footnote{Disponível em
  \url{https://github.com/}.}. Serviço de Web Hosting Compartilhado
para projetos que usam o controle de versionamento e também onde
organizamos as tarefas diárias.  Kile e Latex\footnote{Disponível em
  \url{http://kile.sourceforge.net/}.}. Editor de código
TeX/LateX. Latex, conjunto de macros para o processador de textos.

Para realização do trabalho foram cumpridas as seguintes etapas.

\begin{enumerate}
\item Familiarização com o objeto da pesquisa. Consistiu na leitura da
  literatura sobre contornos, produção de resenhas e sessões de dúvidas
  com o orientador. De 01/01/2012 à 30/01/2012.
\item Alimentação da planilha de operações a partir dos artigos
  analisados. De 23/01/2012 à 06/06/2012.
\item Treinamento com ferramentas Ubuntu, Python e Musicontour, para
  auxiliar no trabalho. De 20/06/2012 à 11/07/2012.
\item Revisão das operações contidas na planilha. De 28/06/2012 à 04/07/2012.
\end{enumerate}

% Marcos. Neste local apenas informe no que consiste a planilha. Tente
% não passar de duas linhas. Fale sobre a necessidade de ser revisada
% e formatada na parte de discussão. Siga esse princípio para as
% outras partes
% Eduardo: Eduardo: Dá uma olhada pra ver se ficou beleza.
A planilha de operações é um catálogo onde estão mapeadas todas as
operações musicais listadas nos artigos estudados.

% Marcos. O que você quer dizer com 'estilizada'? Por que não
% 'formatada'?
% Eduardo: É que eu não tinha encontrado a palavra certa. OK.
Havia uma planilha com algumas operações já inclusas, porém ela
precisava ser revisada e formatada.
% Marcos. De que forma você analisou os artigos? Você na verdade
% extraiu as operações.
% Eduardo: Ok
% Marcos. Prefira sempre a voz ativa e a ordem direta: Eu procurei
% operações de contornos relacionadas com obras da literatura.....
Eu procurei operações de contornos relacionadas com obras da literatura
musical e adicionei à planilha. Concluído esta etapa, bastava
analizar cada operação inclusa na planilha separadamente. Como haviam
muitas operações e algumas necessitavam de cáculos extensos e
demorados, utilizei o software MusiContour. Para que este
software pudesse ser utilizado foi necessário um treinamento para
Python, para o MusiContour e também a instalação do sistema
operacional Ubuntu. Utilizei o Ubuntu para agilizar a instalação
e utilização dos proramas necessários. Como eu utilizava Windows
estávamos perdendo tempo tentando descobrir comandos de instalações e
traduções da linguagem do Ubuntu para Windows. Finalmente para
elaborar o relatório utilizei o software kile, que permite uma
facilidade no trabalho de edição, um aprendizado adicional sobre
programação, e um aprofundamento nos softwares citados em materiais.

\Section{Resultados}
\label{sec:resultados}
\info{Relação dos resultados ou produtos obtidos durante a execução da
  pesquisa, indicando os avanços no conhecimento disponível obtidos
  com a execução da pesquisa.}

% planilha com mapeamento de operações e o que funciona
% aprendizado

% Marcos: Você na verdade tem três resultados principais: o mapeamento
% das operações, os testes das operações e a lista de inconsistências
% que você encontrou. Então fale mais em mapeamento do que em
% planilha. Você pode falar da planilha apenas para a pessoa entender
% como você registrou esse mapeamento e esses testes
% Eduardo: Ok. Consegui distinguir a diferença entre mapeamento e planilha a
% partir da explicação do e-mail.

% Marcos: evite muitos adjetivos no texto. Várias vezes você diz que
% algo é muito importante. No caso dos testes basta dizer o que eles
% são e o que revelam
% Eduardo : ok
O mapeamento das operações foi criado com intuito de servir como uma 
espécie de catálgo de operações. Este catálogo serve para consultas rápidas de
um determinado contorno 'Vide mais informações na seção discussão'.
Após o mapeamento das operações na planilha testei todas as operações que 
nela estavam contidas. Estes testes me levaram a encotrar consistências em 
algumas das operações.
% Marcos: Este é o local mais apropriado para fazer uma descrição da
% planilha. Você pode até inserir uma referência cruzada na seção
% anterior dizendo que há mais informações sobre a planilha nesta
% seção
% Eduardo: Como assim uma referência cruzada?
% Marcos: uma referência cruzada é o simples mencionar que uma
% informação está na outra seção. Por exemplo, 'vide mais informações
% na seção de discussão'

% Marcos: Melhor: "Este trabalho teve como resultados secundários:"
% Eduardo: Ok
Este trabalho teve como resultados secundários:

\begin{enumerate}
\item Aprimoramento da leitura em língua inglesa.
\item Aprofundamento do conhecimento sobre contornos.
\item Aprendizado sobre Ubuntu, Python, Kile, MusiContour.
\item Aprimoramento da escrita de textos acadêmicos.
\item Aprimoramento de organização e técnicas de estudo.
\end{enumerate}

\Section{Discussão}
\label{sec:discussao}
\info{Expor de modo sucinto a contribuição do seu projeto ao projeto
  de pesquisa do orientador e ao conhecimento científico da sua área,
  apresentando as implicações para futuros trabalhos que podem ser
  desenvolvidos.}
% comentar o que foi dito nas outras seções. ir além da descrição

% Marcos: É preciso organizar esta seção. Você pode fazer agrupar as
% informações em discussões e dificuldades.

% Marcos: você pode dizer que as ferramentas que utilizou deu
% agilidade ao trabalho e economizou tempo e energia. Com o GitHub
% você manteve sua lista de tarefas organizada. Vale a pena colocar um
% link para a sua lista de tarefas. Essa é a hora de impressionar seu
% leitor (veja o link mais abaixo). Sobre o MusiContour você pode
% dizer que economizou muito tempo e mostrar como se faz uma matriz de
% comparação. Veja meu texto na dissertação para ter uma ideia de como
% descrever isso. É bom você mostrar a dificuldade para as pessoas
% entenderem como fazer isso gasta tempo.

% https://github.com/msampaio/MusiContour/issues/assigned/EduardoNunes?direction=desc&sort=created&state=closed

Foi necessário utilizar os materiais citados, eu entrei como bolsista
substituto, o que implica que além do tabalho a realizar, ainda havia
o treinamento que reduziria o tempo de trabalho. Materiais como Github
e MusiContour foram utilizados para suprir este tempo que foi
reduzido. O github para a organização das tarefas diárias\footnote{As
  tarefas realizadas estão disponíveis para consulta em
  \url{http://goo.gl/4Ie7c}.}, e o MusiContour que em 1 segundo
operações com matrizes que levariam minutos para serem calculadas.


Cálculo que o Musicontour faz:
% Eduardo: Seria bom colocar um print nessa parte


% Marcos: A organização em tópicos deve estar associada à parte da
% discussão do resultado. É preciso colocar um trecho da planilha em
% anexo. Eu vou cuidar do anexo.
% Eduardo: Certo. Mas vamos assciar o texto ao anexo, ou o anexo vai sere
% associado ao texto?
O mapeamento de operações tem o objetivo de reduzir este atraso em
segundos, já que todas as operações a serem procuradas estão mapeadas
e organizadas nos seguintes tópicos: Artigo, tipo de operação, página,
obra, compositor, texto ou gráfico, parágrafo ou figura.
% Marcos: desnecessário:
% Eduardo: retirado.

A pesquisa me possibilitou um contato direto com diversos artigos que abordavam
o tema contornos. Este tipo de contato que não se encontra no curso de graduação,
me trouxe um conhecimento aprofundado sobre a teoria de relações de contornos muscais.
Me possiblitou também, um aprimoramento na leitura da lígua inglesa, escrita e conscisão
de resenhas, e um conhecimento sobre diversas ferramentas computacionais.

% Marcos: refira-se à etapa que você listou. Prefiro que chame de
% 'teste de operações'. Use a primeira pessoa no singular. Você pode
% dizer estas inconsistências terão posterior análise, pois a
% verificação de seu impacto na Teoria requer tempo
% Eduardo: Ok (esse comentário é refente ao parágrafo seguinte).

% Marcos: É importante listar as operações em que você encontrou as
% inconsistências. Coloque em uma lista enumerada.
% Eduardo: Ok
Durante o teste de operações, encontrei algumas das operações com 
inconsistência no seu resultado. Estas inconsistências terão posteriores 
análises, pois a verificação de seu impacto na teoria requer tempo.
% Eduardo, eu teria que esplicar algo sobre as operações citadas??
Operações com inconsistência.
\begin{enumerate}
\item Friedmann1985, Pg 241/241, Operação CCVII.
\item Quinn, Pg 242, Operação CSIM.
\item Quinn, Pg 262, C+SIM.
\item Schultz2008, Pg 113, Operação CC.
\end{enumerate}

Entrei como bolsista substituto em janeiro. Isso resultou na necessidade
de um treinamento que consumiu um certo tempo. A partir desta necessidade
houve um remanejamento no plano de trabalho. A importância de verificar cada operação
da planilha era maior do que criar o banco de figuras ou os exemplos para o
tutorial online, importância essa que nos mostrou que em algumas operações
haviam inconsistências.

%%%%%%%%% bibliografia %%%%%%%%%%%%%%%%
% insere bibliografia

\renewcommand{\refname}{Referências bibliográficas (máximo 15)}
\info{Relação itemizada das referências que subsidiam a proposta de
  pesquisa, colocando as mais importantes.}

\nocite{
  Friedmann1985,
  Friedmann1987,
  Morris1987,
  Marvin1987,
  Marvin1988,
  Polansky1992,
  Morris1993,
  Clifford1995,
  Quinn1997,
  Beard2003,
  Sampaio2008,
  Schultz2008,
  Schultz2009,
  Bor2009
}

\bibliographystyle{plain}
\bibliography{bibliography}

\Section{Participação em reuniões científicas e publicações}
\info{Relacionar as reuniões científicas e os títulos dos trabalhos
apresentados pelo estudante durante a vigência da bolsa. Incluir
títulos de publicações que resultaram ou se beneficiaram de seu
trabalho.}

\Section{Anexos}
\info{Anexar os resumos ou trabalhos que foram apresentados pelo bolsista
durante a vigência da bolsa.}

\end{document}
