%% Local IspellDict: brasileiro

\documentclass[11pt]{article}
\usepackage{relatorio-pibic}
\usepackage[utf8x]{inputenc}

% usar para palavras estrangeiras: \eng{english word}
\newcommand{\eng}[1]{\textit{#1}}

\begin{document}

\graphicspath{{figs/}}

\cabecalho{Relatório Final}

\dadosRelatorioFinal
{Levantamento de operações de teorias de contornos em análises de
  obras musicais}
{Verificando inconsistências nas teorias de contornos musicais através
  de ferramentas computacionais }
{Eduardo Lago Nunes}
{Marcos da Silva Sampaio}
{Genos}
{Inserir 3 palavras chave}
{JANEIRO A JULHO DE 2012}

\resumo{Inserir texto do resumo}

\newpage

\setcounter{page}{1}
\onehalfspace

% Marcos: orientações gerais:

% Ao inserir notas coloque seu nome, como fiz com o meu. Apague cada
% nota à medida que as implementar para manter o código limpo.

% Inicialmente não se preocupe tanto com o texto, mas com os dados
% para usar no texto. Depois que tiver a lista dos materiais, das
% etapas cumpridas, dos resultados, das dificuldades, você pode
% refletir sobre cada um e escrever anotações sobre eles. Em seguida
% você vai transformando essas anotações no seu texto

% Foque primeiro nos materiais e métodos. O que você fez, como você
% fez. Arrume esta seção antes das outras

% O seu leitor pode não entender nada de contornos. Com o seu texto
% ela precisa entender que havia um problema para ser
% resolvido---simplificar os testes e verificação de operações de
% contornos---e você resolveu com a planilha

% Não use vírgula entre sujeito e verbo

\Section{Introdução}
\label{sec:introducao}
\info{Delimitação do problema trabalhado e as conexões entre o plano
  de trabalho do bolsista e o projeto do orientador. Objetivos e
  justificativa do plano em termos de relevância para a pesquisa
  cientifica e do estado da arte.}
% importância de contornos para música, da teoria de contornos, a
% inconsistência...

% Marcos: O que são contornos? Você sempre pode parafrasear alguém e
% citar.
Contornos são importantes pelo fato de fazer ser mais perceptível uma
certa melodia ou ritmo que esteja sendo executado, até mesmo para
pessoas com ouvido destreinado.

% Marcos: Qual a sua ideia neste parágrafo? Qual ferramenta? De quem é
% a teoria?

% Marcos: Remover vírgula entre sujeito e verbo
A teoria de contornos, visa estudar esta ferramenta, com intuito de
desenvolver técnicas para análises do estudo dos padrões e
comportamento destes contornos.
% Marcos: quem vai desenvolver? Que pesquisas são essas? É o projeto
% do PIBIC? É outra coisa?
A partir destes estudos, desenvolver uma solução para uma
inconsistência encontrada durante as pesquisas do orientador.

% nas seções seguintes descrever as coisas
\Section{Materiais e métodos}
\label{sec:materiais}
\info{Descrição da maneira como foram desenvolvidas as atividades para
  se chegar aos objetivos propostos. Indicar o material e métodos que
  foram usados.}
% materiais: artigos, MusiContour (dizer que faz parte da pesquisa), planilha
% métodos: como fez cada coisa

% Marcos: Comece com algum texto como "Neste trabalho utilizei como materiais
% os artigos....". Coloque uma nota de rodapé com um link para cada
% programa. Siga o modelo de MusiContour

% Eduardo: acho que falta algum, só não lembro.
% Marcos: Veja os nossos emails, o GitHub e suas anotações para listar
% todos os materiais que você usou.
Os artigos sobre teoria dos contornos, os programas de computador:
Kile, MusiContour\footnote{Disponível em
  \url{http://genosmus.com/MusiContour}.}, Python e Git, e a planilha
de operações.

% Marcos: Para sua organização acho melhor você fazer uma lista
% primeiro. Assim você vai organizar melhor suas ideias. Em seguida
% você tenta descrever brevemente cada uma das etapas. Depois você
% transforma a lista em um texto corrido. Você pode usar esta
% estrutura abaixo para fazer uma lista enumerada:

\begin{enumerate}
\item Leitura de artigos
\item Catalogação de operações
\end{enumerate}

% Marcos: Reveja a digitação de MusiContour e insira n

% Eduardo: Como eu vou falar do GIT e do KILE nessa parte?
% Marcos: Sobre Git e Kile, fale no final, como ferramentas para a
% redação do relatório final
Primeiramente a leitura juntamente com a catalogação das operações dos
artigos propostos pelo orientador, à medida em que lia, catalogava na
planilha de operações.
% Marcos: Fale sobre a linguagem de programação e quem desenvolveu o
% MusiContour na parte em que você fala apenas sobre materiais. Nesta
% parte fale apenas sobre a metodologia, em como você usou o
% MusiContour, e não como o programa é.
Em seguida a análise de cada operação isoladamente, utilizando a
ferramete MusiCountour, desenvolvida no Python por Marcos da Silva.

\Section{Resultados}
\label{sec:resultados}
\info{Relação dos resultados ou produtos obtidos durante a execução da
  pesquisa, indicando os avanços no conhecimento disponível obtidos
  com a execução da pesquisa.}

% planilha com mapeamento de operações e o que funciona
% aprendizado


% Marcos: Sugiro fazer uma lista dos produtos e depois escrever
% brevemente sobre cada item. A pesquisa te possibilitou o contato com
% os artigos, não a planilha de operações

% Marcos: Corrija a vírgula entre o sujeito e o verbo, a digitação e
% ortografia das palavras. Prefira períodos curtos, então substitua
% 'artigos, contato este...' por 'artigos. Este contato...'
A planilha de operações, me possibilitou um contato direto com
diversos artigos, contato este que eu não teria no curso de graduação,
trazendo assim, um conhecimento aprofundado da teoria de relações de
contornos muscais. Me possiblitou também, um aprimoramento na leitura
da lígua inglesa e um aprimoramento na escrita e conscisão de
resenhas.

\Section{Discussão}
\label{sec:discussao}
\info{Expor de modo sucinto a contribuição do seu projeto ao projeto
  de pesquisa do orientador e ao conhecimento científico da sua área,
  apresentando as implicações para futuros trabalhos que podem ser
  desenvolvidos.}

% comentar o que foi dito nas outras seções. ir além da descrição

% Marcos: remover vírgula separando sujeito do verbo. Você pode
% reescrever melhor este parágrafo, separando em períodos curtos e
% focando na agilidade de encontrar operações com mínimo tempo e
% esforço.
Esta pesquisa, tem sua importância em reduzir significativamente o
tempo de trabalho do orientador, sendo que, ele terá em mãos todas as
operações conferidas e mapeadas em uma planilha, agilizando a procurar
para o encontro destas operações.

% Marcos: O parágrafo está solto. O 'ainda assim' pressupõe que a
% ideia deste parágrafo é adversativa a algum outro, mas não há outro
% parágrafo.

% Marcos: Prefira sempre a ordem direta das frases e evite a voz
% passiva. Diga que você encontrou incoerências após testar as
% operações. Substitua o "devidas providências" por algo como
% "posterior avaliação"
Ainda assim, após a análise detalhada de cada operação em particular,
foram encontradas incoerências, estas serão estudadas para que sejam
tomadas as devidas providências.

% Marcos: Não coloque a bibliografia por extenso aqui. Eu irei
% formatar a bibliografia automaticamente depois. Se quiser
% acrescentar algum livro, apenas coloque o nome do autor e o ano como
% um comentário, dessa forma:

% Morris 1993

%%%%%%%%% bibliografia %%%%%%%%%%%%%%%%
% insere bibliografia

\renewcommand{\refname}{Referências bibliográficas (máximo 15)}
\info{Relação itemizada das referências que subsidiam a proposta de
  pesquisa, colocando as mais importantes.}

\nocite{Morris1993,Bor2009,Friedmann1985,Friedmann1987,Marvin1988,Marvin1987,Morris1993,Sampaio2008}

\bibliographystyle{plain}
\bibliography{bibliography}

\Section{Participação em reuniões científicas e publicações}
\info{Relacionar as reuniões científicas e os títulos dos trabalhos
apresentados pelo estudante durante a vigência da bolsa. Incluir
títulos de publicações que resultaram ou se beneficiaram de seu
trabalho.}

\Section{Anexos}
\info{Anexar os resumos ou trabalhos que foram apresentados pelo bolsista
durante a vigência da bolsa.}

\end{document}
