\documentclass[11pt]{article}
\usepackage{relatorio-pibic}
\usepackage[utf8x]{inputenc}

% usar para palavras estrangeiras: \eng{english word}
\newcommand{\eng}[1]{\textit{#1}}

\begin{document}

\graphicspath{{figs/}}

\cabecalho{Relatório Final}

\dadosRelatorioFinal
{Levantamento de operações de teorias de contornos em análises de
  obras musicais}
{Verificando inconsistências nas teorias de contornos musicais através
  de ferramentas computacionais }
{Eduardo Lago Nunes}
{Marcos da Silva Sampaio}
{Genos}
{Inserir 3 palavras chave}
{JANEIRO A JULHO DE 2012}

\resumo{Inserir texto do resumo}

\newpage

\setcounter{page}{1}
\onehalfspace

\Section{Introdução}
\label{sec:introducao}
\info{Delimitação do problema trabalhado e as conexões entre o plano
  de trabalho do bolsista e o projeto do orientador. Objetivos e
  justificativa do plano em termos de relevância para a pesquisa
  cientifica e do estado da arte.}

% importância de contornos para música, da teoria de contornos, a
% inconsistência...

Contornos são importantes pelo fato de fazer ser mais perceptível uma
certa melodia ou ritmo que esteja sendo executado, até mesmo para
pessoas com ouvido destreinado.

A teoria de contornos, visa estudar esta ferramenta, com intuito de
desenvolver técnicas para análises do estudo dos padrões e
comportamento destes contornos.  A partir destes estudos, desenvolver
uma solução para uma inconsistência encontrada durante as pesquisas do
orientador.


% nas seções seguintes descrever as coisas
\Section{Materiais e métodos}
\label{sec:materiais}
\info{Descrição da maneira como foram desenvolvidas as atividades para
  se chegar aos objetivos propostos. Indicar o material e métodos que
  foram usados.}

% materiais: artigos, MusiContour (dizer que faz parte da pesquisa), planilha
% métodos: como fez cada coisa

% Eduardo: acho que falta algum, só não lembro.
Os artigos sobre teoria dos contornos, os programas de computador:
Kile, MusiContour, Python e Git, e a planilha de
operações.

% Eduardo: Como eu vou falar do GIT e do KILE nessa parte?
Primeiramente a leitura juntamente com a catalogação das operações dos
artigos propostos pelo orientador, à medida em que lia, catalogava na
planilha de operações.  Em seguida a análise de cada operação
isoladamente, utilizando a ferramete MusiCountour, desenvolvida no
Python por Marcos da Silva.

MusiContour\footnote{Disponível em \url{http://genosmus.com/MusiContour}.}

\Section{Resultados}
\label{sec:resultados}
\info{Relação dos resultados ou produtos obtidos durante a execução da
  pesquisa, indicando os avanços no conhecimento disponível obtidos
  com a execução da pesquisa.}

% planilha com mapeamento de operações e o que funciona
% aprendizado

A planilha de operações, me possibilitou um contato direto com
diversos artigos, contato este que eu não teria no curso de graduação,
trazendo assim, um conhecimento aprofundado da teoria de relações de
contornos muscais. Me possiblitou também, um aprimoramento na leitura
da lígua inglesa e um aprimoramento na escrita e conscisão de
resenhas.

\Section{Discussão}
\label{sec:discussao}
\info{Expor de modo sucinto a contribuição do seu projeto ao projeto
  de pesquisa do orientador e ao conhecimento científico da sua área,
  apresentando as implicações para futuros trabalhos que podem ser
  desenvolvidos.}

% comentar o que foi dito nas outras seções. ir além da descrição
Esta pesquisa, tem sua importância em reduzir significativamente o
tempo de trabalho do orientador, sendo que, ele terá em mãos todas as
operações conferidas e mapeadas em uma planilha, agilizando a procurar
para o encontro destas operações.

Ainda assim, após a análise detalhada de cada operação em particular,
foram encontradas incoerências, estas serão estudadas para que sejam
tomadas as devidas providências.


%%%%%%%%% bibliografia %%%%%%%%%%%%%%%%
% insere bibliografia

Bor, Mustafa. Contour reduction algorithms : a theory of pitch and duration hierarchies for post-tonal music.Phd thesis, The University of British Columbia, Vancouver, 2009.
Friedmann, Michael L. A methodology for the discussion of contour: its application to schoenberg’s music.Journal of Music Theory, 29(2):223-48, 1985.
Friedmann, Michael L. A response: My contour, their contour.Journal of Music Theory, 31(2):268-274, 1987.
Marvin, Elizabeth Wes. A generalized theory of musical contour: its application to melodic and rhythmic analysis of non-tonal music and its perceptual and pedagogical implications.PhD thesis, University of Rochester, 1988.
Marvin, Elizabeth West and Laprade, Paul A.. Relating musical contours: Extensions of a theory for contour.Journal of Music Theory, 31(2):225-67, 1987.
Morris, Robert Daniel. Composition with Pitch-classes: A Theory of Compositional Design.Yale University Press, 1987.
Sampaio, Marcos da Silva. Em torno da romã: aplicações de operações de contornos na composição.Master’s thesis, Universidade Federal da Bahia, Salvador, BA, 2008.

\renewcommand{\refname}{Referências bibliográficas (máximo 15)}
\info{Relação itemizada das referências que subsidiam a proposta de
  pesquisa, colocando as mais importantes.}

\nocite{Morris1993}

\bibliographystyle{plain}
\bibliography{bibliography}

\Section{Participação em reuniões científicas e publicações}
\info{Relacionar as reuniões científicas e os títulos dos trabalhos
apresentados pelo estudante durante a vigência da bolsa. Incluir
títulos de publicações que resultaram ou se beneficiaram de seu
trabalho.}

\Section{Anexos}
\info{Anexar os resumos ou trabalhos que foram apresentados pelo bolsista
durante a vigência da bolsa.}

\end{document}
