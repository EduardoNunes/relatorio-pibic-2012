%% Local IspellDict: brasileiro

\documentclass[11pt]{article}
\usepackage{relatorio-pibic}
\usepackage[utf8x]{inputenc}
\usepackage{cite}

% usar para palavras estrangeiras: \eng{english word}
\newcommand{\eng}[1]{\textit{#1}}

\begin{document}

\graphicspath{{figs/}}

\cabecalho{Relatório Final}

\dadosRelatorioFinal
{Levantamento de operações de teorias de contornos em análises de
  obras musicais}
{Verificando inconsistências nas teorias de contornos musicais através
  de ferramentas computacionais }
{Eduardo Lago Nunes}
{Marcos da Silva Sampaio}
{Genos}
{Inserir 3 palavras chave}
{JANEIRO A JULHO DE 2012}

\resumo{Inserir texto do resumo}

\newpage

\setcounter{page}{1}
\onehalfspace

% Marcos: orientações gerais:

% Inicialmente não se preocupe tanto com o texto, mas com os dados
% para usar no texto. Depois que tiver a lista dos materiais, das
% etapas cumpridas, dos resultados, das dificuldades, você pode
% refletir sobre cada um e escrever anotações sobre eles. Em seguida
% você vai transformando essas anotações no seu texto

% Foque primeiro nos materiais e métodos. O que você fez, como você
% fez. Arrume esta seção antes das outras

% O seu leitor pode não entender nada de contornos. Com o seu texto
% ela precisa entender que havia um problema para ser
% resolvido---simplificar os testes e verificação de operações de
% contornos---e você resolveu com a planilha

% Não use vírgula entre sujeito e verbo

\Section{Introdução}
\label{sec:introducao}
\info{Delimitação do problema trabalhado e as conexões entre o plano
  de trabalho do bolsista e o projeto do orientador. Objetivos e
  justificativa do plano em termos de relevância para a pesquisa
  cientifica e do estado da arte.}
% importância de contornos para música, da teoria de contornos, a
% inconsistência...

% Marcos: O que são contornos? Você sempre pode parafrasear alguém e
% citar.
Contornos são importantes pelo fato de fazer ser mais perceptível uma
certa melodia ou ritmo que esteja sendo executado, até mesmo para
pessoas com ouvido destreinado.

% Marcos: Qual a sua ideia neste parágrafo? Qual ferramenta? De quem é
% a teoria?

% Marcos: Remover vírgula entre sujeito e verbo
A teoria de contornos visa estudar esta ferramenta com intuito de
desenvolver técnicas para análises do estudo dos padrões e
comportamento destes contornos.
% Marcos: quem vai desenvolver? Que pesquisas são essas? É o projeto
% do PIBIC? É outra coisa?
A partir destes estudos, desenvolver uma solução para uma
inconsistência encontrada durante as pesquisas do orientador.

% nas seções seguintes descrever as coisas
\Section{Materiais e métodos}
\label{sec:materiais}
\info{Descrição da maneira como foram desenvolvidas as atividades para
  se chegar aos objetivos propostos. Indicar o material e métodos que
  foram usados.}

  \begin{enumerate} 
\item Leitura de artigos.
\item Catalogação de operações.
\item Revisão com MusiContour.
\item Preparo do relatório.
\end{enumerate}

%materiais: EDUARDO: como eu falaria sobre o git?
Neste trabalho utilizei como materiais os artigos sobre teoria de contornos, Mendeley
que é o local onde estavam armazenados os arquivos, Musicontour que é uma ferramenta 
de análise de contornos desenvolvida por Marcos da Silva (orientador), Python que é um 
programa de plataforma onde foi desenvolvido o musicontour, Linux que foi o  sistema 
operacional mais adequado para pesquisa, Git, Github que é um site onde organizamos 
todas as tarefas diárias, Kile que é um editor de relatório, latex que é uma espécie 
de compilador para o kile, e a planilha de operações que é o resultado do trabalho.

%Métodos: EDUARDO: estou escrevendo tudo para que formatemos depois, OBS.: Faltou falar 
%sobre o GIT, pq eu não sei o que falar.
A planilha de operações, apresenta todas as operações inclusas na literatura musical
que foram citadas nos artigos estudados. Após a leitura de todos os artigos e inclusão das
operações na planilha, foi necessário uma revisão em cada operação específica contida 
na planilha, para esta revisão foi utilizado o programa Musicontour, com intuito de agilizar
o processo, já que muitas das operações possuem cálculos extensos e que levariam muito mais
tempo para serem revisadas. Para que o Musicountour funcionasse adequadamente, foi necessário
instalar o sistema operacional Linux (que é sistema ultilizado pelo orientador) para facilitar
e agilizar o trabalho de ter que descobrir cada comando no sistema operacional utilizado pelo
orientando, e o programa python, que é a plataforma para que o musicontour funcione 
adequadamente. 
O Github, é um site onde eram listadas as tarefas para uma melhor organização do trabalho.
Após a conclusão da planilha e revisão das operações, utilizamos o programa Kile para edição
do relatório, após sua edição ele seria compilado no Latex, que já apresenta o relatório
na formatação correta.


%EDUARDO achei embolado essa parte...
Materiais utilizados: Mendeley\footnote{Disponível em
  \url{http://www.mendeley.com/}.}, Artigos\footnote{Disponível em
  \url{http://www.mendeley.com/groups/1138861/_/papers/}.},
  MusiContour\footnote{Disponível em \url{http://genosmus.com/MusiContour}.},
  Python\footnote{Disponível em  \url{http://www.python.org/getit/}.},
  Linux\footnote{Disponível em
  \url{http://www.ubuntu.com/download/desktop}.},
  Git\footnote{Disponível em
  \url{http://git-scm.com/downloads}.}, Github\footnote{Disponível em
  \url{https://github.com/}.}, Kile e Latex\footnote{Disponível em
  \url{http://kile.sourceforge.net/}.} e planilha de operações.



\Section{Resultados}
\label{sec:resultados}
\info{Relação dos resultados ou produtos obtidos durante a execução da
  pesquisa, indicando os avanços no conhecimento disponível obtidos
  com a execução da pesquisa.}

% planilha com mapeamento de operações e o que funciona
% aprendizado


  \begin{enumerate} 
\item Contato direto com os artigos.
\item Conhecimento aprofundado sobre contornos.
\item Aprimoramento na leitura em inglês.
\item Escrita e consisão de resenhas.
\item Utilização de ferramentas.
\end{enumerate}

A pesquisa me possibilitou um contato direto com diversos artigos que abordavam
o tema contornos. Este tipo de contato que não se encontra no curso de graduação, 
me trouxe um conhecimento aprofundado sobre a teoria de relações de contornos muscais. 
Me possiblitou também, um aprimoramento na leitura da lígua inglesa, escrita e conscisão 
de resenhas, e um conhecimento sobre diversas ferramentas computacionais.



\Section{Discussão}
\label{sec:discussao}
\info{Expor de modo sucinto a contribuição do seu projeto ao projeto
  de pesquisa do orientador e ao conhecimento científico da sua área,
  apresentando as implicações para futuros trabalhos que podem ser
  desenvolvidos.}

% comentar o que foi dito nas outras seções. ir além da descrição


Esta pesquisa tem sua importância em reduzir significativamente o
tempo de trabalho do orientador. Como todas as operações estarão mapeadas
na planilha, basta olhar onde se localiza e como está descrita a operação
desejada. O catálogo indica o artigo, a página, o nome da obra, o compositor,
se está em gráfico ou texto e em que parágrafo ou figura. Assim, basta escolhar 
uma operação e ir diretamente no artigo, pagina e parágrafo indicado pelo
catálogo.
% Marcos: O parágrafo está solto. O 'ainda assim' pressupõe que a
% ideia deste parágrafo é adversativa a algum outro, mas não há outro
% parágrafo.

% Marcos: Prefira sempre a ordem direta das frases e evite a voz
% passiva. Diga que você encontrou incoerências após testar as
% operações. Substitua o "devidas providências" por algo como
% "posterior avaliação"
Ainda assim, após a análise detalhada de cada operação em particular,
foram encontradas incoerências, estas serão estudadas para que sejam
tomadas as devidas providências.

% Marcos: Não coloque a bibliografia por extenso aqui. Eu irei
% formatar a bibliografia automaticamente depois. Se quiser
% acrescentar algum livro, apenas coloque o nome do autor e o ano como
% um comentário, dessa forma:

% Morris 1993

%%%%%%%%% bibliografia %%%%%%%%%%%%%%%%
% insere bibliografia

\renewcommand{\refname}{Referências bibliográficas (máximo 15)}
\info{Relação itemizada das referências que subsidiam a proposta de
  pesquisa, colocando as mais importantes.}

\nocite{Morris1993,Bor2009,Friedmann1985,Friedmann1987,Marvin1988,Marvin1987,Morris1993,Sampaio2008}

\bibliographystyle{plain}
\bibliography{bibliography}

\Section{Participação em reuniões científicas e publicações}
\info{Relacionar as reuniões científicas e os títulos dos trabalhos
apresentados pelo estudante durante a vigência da bolsa. Incluir
títulos de publicações que resultaram ou se beneficiaram de seu
trabalho.}

\Section{Anexos}
\info{Anexar os resumos ou trabalhos que foram apresentados pelo bolsista
durante a vigência da bolsa.}

\end{document}
