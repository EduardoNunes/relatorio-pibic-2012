\documentclass[11pt]{article}
\usepackage{relatorio-pibic}
\newcommand{\eng}[1]{\textit{#1}}
\usepackage[utf8x]{inputenc}

\begin{document}

\graphicspath{{figs/}}

\cabecalho{Relatório Final}

\dadosRelatorioFinal
{Levantamento de operações de teorias de contornos em análises de
  obras musicais}
{Verificando inconsistências nas teorias de contornos musicais através
  de ferramentas computacionais }
{Eduardo Lago Nunes}
{Marcos da Silva Sampaio}
{Genos}
{Inserir 3 palavras chave}
{JANEIRO A JULHO DE 2012}

\resumo{Inserir texto do resumo}

\newpage

\setcounter{page}{1}
\onehalfspace

\Section{Introdução}
\label{sec:introducao}
\info{Delimitação do problema trabalhado e as conexões entre o plano
  de trabalho do bolsista e o projeto do orientador. Objetivos e
  justificativa do plano em termos de relevância para a pesquisa
  cientifica e do estado da arte.}

% importância de contornos para música, da teoria de contornos, a inconsistência...


% nas seções seguintes descrever as coisas
\Section{Materiais e métodos}
\label{sec:materiais}
\info{Descrição da maneira como foram desenvolvidas as atividades para se
chegar aos objetivos propostos. Indicar o material e métodos que foram
usados.}

% materiais: artigos, MusiContour (dizer que faz parte da pesquisa), planilha
% métodos: como fez cada coisa

MusiContour\footnote{Disponível em \url{http://genosmus.com/MusiContour}.}

\Section{Resultados}
\label{sec:resultados}
\info{Relação dos resultados ou produtos obtidos durante a execução da
pesquisa, indicando os avanços no conhecimento disponível obtidos com
a execução da pesquisa.}
% planilha com mapeamento de operações e o que funciona
% aprendizado

\Section{Discussão}
\label{sec:discussao}
\info{Expor de modo sucinto a contribuição do seu projeto ao projeto de
pesquisa do orientador e ao conhecimento científico da sua área,
apresentando as implicações para futuros trabalhos que podem ser
desenvolvidos.}
% comentar o que foi dito nas outras seções. ir além da descrição

%%%%%%%%% bibliografia %%%%%%%%%%%%%%%%
\renewcommand{\refname}{Referências bibliográficas  (máximo 15)}
\info{Relação itemizada das referências que subsidiam a proposta de
pesquisa, colocando as mais importantes.}

% insere bibliografia

\nocite{Morris1993}

\bibliographystyle{plain}
\bibliography{bibliography}

\Section{Participação em reuniões científicas e publicações}
\info{Relacionar as reuniões científicas e os títulos dos trabalhos
apresentados pelo estudante durante a vigência da bolsa. Incluir
títulos de publicações que resultaram ou se beneficiaram de seu
trabalho.}

\Section{Anexos}
\info{Anexar os resumos ou trabalhos que foram apresentados pelo bolsista
durante a vigência da bolsa.}

\end{document}
