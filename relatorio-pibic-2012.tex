%% Local IspellDict: brasileiro

\documentclass[11pt]{article}
\usepackage{relatorio-pibic}
\usepackage[utf8x]{inputenc}
\usepackage{cite}

% usar para palavras estrangeiras: \eng{english word}
\newcommand{\eng}[1]{\textit{#1}}

\begin{document}

\graphicspath{{figs/}}

\cabecalho{Relatório Final}

\dadosRelatorioFinal
{Levantamento de operações de teorias de contornos em análises de
  obras musicais}
{Verificando inconsistências nas teorias de contornos musicais através
  de ferramentas computacionais }
{Eduardo Lago Nunes}
{Marcos da Silva Sampaio}
{Genos}
{Teoria de Relações de Contornos Musicais, Teoria Musical, Computação Musical}
{JANEIRO A JULHO DE 2012}

\resumo{Inserir texto do resumo}

\newpage

\setcounter{page}{1}
\onehalfspace

% Marcos: orientações gerais:

% Inicialmente não se preocupe tanto com o texto, mas com os dados
% para usar no texto. Depois que tiver a lista dos materiais, das
% etapas cumpridas, dos resultados, das dificuldades, você pode
% refletir sobre cada um e escrever anotações sobre eles. Em seguida
% você vai transformando essas anotações no seu texto

% Foque primeiro nos materiais e métodos. O que você fez, como você
% fez. Arrume esta seção antes das outras

% O seu leitor pode não entender nada de contornos. Com o seu texto
% ele precisa entender que havia um problema para ser
% resolvido---simplificar os testes e verificação de operações de
% contornos---e você resolveu com a planilha
% Ao invés de focar em falar da planilha, fale do mapeamento de
% operações

% Não use vírgula entre sujeito e verbo

\Section{Introdução}
\label{sec:introducao}
\info{Delimitação do problema trabalhado e as conexões entre o plano
  de trabalho do bolsista e o projeto do orientador. Objetivos e
  justificativa do plano em termos de relevância para a pesquisa
  cientifica e do estado da arte.}
% importância de contornos para música, da teoria de contornos, a
% inconsistência...

% Marcos: O que são contornos? Você sempre pode parafrasear alguém e
% citar.
% Eduardo: Ok
Contornos são elementos organizados numericamente do mais baixo para
o mais alto sendo que deve-se ignorar o exato intervalo entre eles.
``Marvin 1987 - Relating Musical Contours: Extensions of a Theory for Contour,
Página 226''.
% Marcos: Qual a sua ideia neste parágrafo? Qual ferramenta? De quem é
% a teoria?
% Eduardo: Ok
Estudos provam que contornos são importantes pelo fato de fazer ser 
mais perceptível uma certa melodia ou ritmo que esteja sendo executado, 
até mesmo para pessoas com ouvido destreinado.

% Marcos: Remover vírgula entre sujeito e verbo
% Eduardo: Ok
Pesquisadores criaram a teoria de contornos visando o estudo dos padrões e
comportamento destes contornos.
% Marcos: quem vai desenvolver? Que pesquisas são essas? É o projeto
% do PIBIC? É outra coisa?

Através de estudos o orientador descobriu uma incosistência em uma das teorias.
Para facilitar sua pesquisa foi desenvolvido um trabalho secundário que é a
criação da planilha de operações.

% Eduardo: passei essa parte de comentário pra cima por se tratar da questão anterior
% Marcos: pense no objetivo geral do seu plano de trabalho para ter
% mais facilidade de escrever este trecho: "Levantamento de operações
% de teorias de contornos em análises de obras musicais". A planilha é
% um recurso que você usou para fazer este levantamento de
% operações.

% nas seções seguintes descrever as coisas
\Section{Materiais e métodos}
\label{sec:materiais}
\info{Descrição da maneira como foram desenvolvidas as atividades para
  se chegar aos objetivos propostos. Indicar o material e métodos que
  foram usados.}

% Marcos: corrigir digitação. Melhor: '...usamos os seguintes
% materiais:'
Para o desenvolvimento da planilha de operações houve a necessidade de
utilizarmos diversos materias.

% Marcos: Siga o modelo. Descreva cada item em uma ou duas linhas

\begin{enumerate}
\item Planilha de mapeamento de operações de contornos. Esta planilha
  contém campos [preencher]. Vide mais informações na seção de
  resultados.
\item Mendeley\footnote{Disponível em
    \url{http://www.mendeley.com/}}. Repositório colaborativo dos
  textos da pesquisa. Este programa foi necessário para o
  compartilhamento da bibliografia.
\item MusiContour\footnote{Disponível em
    \url{http://genosmus.com/MusiContour}.}. Software de processamento
  de operações de contornos desenvolvido pelo orientador. Esta
  ferramenta foi necessária para realizar testes de operações de
  contornos.
\end{enumerate}

Material da pesquisa que foi analizado para a extração dos contornos.
% Marcos: SEMPRE coloque o mais importante em primeiro lugar
% Eduardo: Ok
Planilha de operações. É o principal material e também o
objetivo do trabalho.
MusiContour\footnote{Disponível em
  \url{http://genosmus.com/MusiContour}.}. Software desenvolvido pelo
orientador que faz os cáculos das operações de contornos.
Python\footnote{Disponível em
  \url{http://www.python.org/getit/}.}. Editor e linguagem de
programação em que foi desenvolvido o software MusiContour.
Linux\footnote{Disponível em
  \url{http://www.ubuntu.com/download/desktop}.}. Linux sistema
operacional utilizado durante a pesquisa.  Git\footnote{Disponível em
  \url{http://git-scm.com/downloads}.}. Sistema de controle de versão
distribuído com ênfase em velocidade.  Github\footnote{Disponível em
  \url{https://github.com/}.}. Serviço de Web Hosting Compartilhado
para projetos que usam o controle de versionamento e também onde
organizamos as tarefas diárias.  Kile e Latex\footnote{Disponível em
  \url{http://kile.sourceforge.net/}.}. Editor de código
TeX/LateX. Latex, conjunto de macros para o processador de textos.


% Marcos: melhor: '...cumpridas as seguintes etapas:'
% Eduardo: Ok
Para realização do trabalho foram cumpridas as seguintes etapas.

% Marcos: Coloque uma ou duas linhas em cada item para explicar o que
% é. Vale a pena indicar o período de cada etapa entre
% parênteses. Siga o modelo com o texto mais abaixo mantendo apenas a
% descrição de ferramentas. Você vai explicar as vantagens das
% ferramentas ou dificuldades na seção de discussão
% Eduardo: Indicar o período de cada etapa é dizer o tempo que cada uma
% durou? Se for devo colocar somente o que foi feito a partir de janeiro né?
\begin{enumerate}
  % Marcos: Você depois fala em treinamento novamente. Melhor chamar
  % esta parte de treinamento inicial ou algo como 'familiarização com
  % o objeto da pesquisa e a teoria de contornos'
  % Eduardo: Ok
\item Familiarização com o objeto da pesquisa. Consistiu na leitura da 
  literatura sobre contornos, produção de resenhas e sessões de dúvidas 
  com o orientador. De 01/01/2012 à 30/01/2012.
\item Alimentação da planilha de operações a partir dos artigos 
  analisados. De 23/01/2012 à 06/06/2012.
\item Treinamento com ferramentas Ubuntu, Python e Musicontour, para
  auxiliar no trabalho. De 20/06/2012 à 11/07/2012.
\item Revisão das operações contidas na planilha. De 28/06/2012 à 04/07/2012.
\end{enumerate}

% Marcos. Neste local apenas informe no que consiste a planilha. Tente
% não passar de duas linhas. Fale sobre a necessidade de ser revisada
% e formatada na parte de discussão. Siga esse princípio para as
% outras partes
% Eduardo: Eduardo: Dá uma olhada pra ver se ficou beleza.
A planilha de operações é um catálogo onde estão mapeadas todas as
operações musicais listadas nos artigos estudados.

% Marcos. O que você quer dizer com 'estilizada'? Por que não
% 'formatada'?
% Eduardo: É que eu não tinha encontrado a palavra certa.
Havia uma planilha com algumas operações já inclusas, porém ela
precisava ser revisada e formatada.
% Marcos. De que forma você analisou os artigos? Você na verdade
% extraiu as operações.
% Eduardo: Ok
% Marcos. Prefira sempre a voz ativa e a ordem direta: Eu procurei
% operações de contornos relacionadas com obras da literatura.....
Eu procurei operações de contornos relacionadas com obras da literatura
musical e adicionei à planilha. Concluído esta etapa, bastava
analizar cada operação inclusa na planilha separadamente. Como haviam
muitas operações e algumas necessitavam de cáculos extensos e
demorados, foi utilizado o software MusiContour. Para que este
software pudesse ser utilizado era necessário um treinamento para
Python, para o MusiContour e também a instalação do sistema
operacional Ubuntu. O Ubuntu foi utilizado para agilizar a instalação
e utilização dos proramas necessários. Como eu utilizava Windows
estávamos perdendo tempo tentando descobrir comandos de instalações e
traduções de linguagem do Ubuntu para Windows. Finalmente para
elaborar o relatório utilizei o software kile, que permite uma
facilidade no trabalho de ediçã, um aprendizado adicional sobre
programação, e um aprofundamento nos softwares citados em materiais.

\Section{Resultados}
\label{sec:resultados}
\info{Relação dos resultados ou produtos obtidos durante a execução da
  pesquisa, indicando os avanços no conhecimento disponível obtidos
  com a execução da pesquisa.}

% planilha com mapeamento de operações e o que funciona
% aprendizado

% Marcos: Você na verdade tem três resultados principais: o mapeamento
% das operações, os testes das operações e a lista de inconsistências
% que você encontrou. Então fale mais em mapeamento do que em
% planilha. Você pode falar da planilha apenas para a pessoa entender
% como você registrou esse mapeamento e esses testes
% Eduardo: Marcos, o mapeamento não seria uma ação e a planilha o resultado??
% Não consigo imaginar o mapeamento como como um resultado.

A partir do mapeamento das operações obtivemos a planilha de operações. 
Esta planilha foi criada com intuito de servir como uma espécie de 
catálgo de operações. Este catálogo serve para consultas rápidas de
um determinado contorno. Os testes das operações também tiveram grande 
importância, pois a partir deles foi possível encontrar uma lista de 
inconsistências.
% Marcos: Este é o local mais apropriado para fazer uma descrição da
% planilha. Você pode até inserir uma referência cruzada na seção
% anterior dizendo que há mais informações sobre a planilha nesta
% seção
% Eduardo: Como assim uma referência cruzada?

% Marcos: Melhor: "Este trabalho teve como resultados secundários:"
% Eduardo: Ok
Este trabalho teve como resultados secundários:

\begin{enumerate}
\item Aprimoramento da leitura em língua inglesa.
\item Aprofundamento do conhecimento sobre contornos.
\item Aprendizado sobre Ubuntu, Python, Kile, MusiContour.
\item Aprimoramento da escrita de textos acadêmicos.
\item Aprimoramento de organização e técnicas de estudo.
\end{enumerate}

\Section{Discussão}
\label{sec:discussao}
\info{Expor de modo sucinto a contribuição do seu projeto ao projeto
  de pesquisa do orientador e ao conhecimento científico da sua área,
  apresentando as implicações para futuros trabalhos que podem ser
  desenvolvidos.}
% comentar o que foi dito nas outras seções. ir além da descrição

% Marcos: É preciso organizar esta seção. Você pode fazer agrupar as
% informações em discussões e dificuldades.
% Dá uma olhada agora.

% Marcos: qual passo? Sugiro começar com algo como: as principais
% dificuldades encontradas neste trabalho foram A, B, C e D. E
% escrever um pouco sobre cada pensando sempre em dizer a razão da
% dificuldade e a solução.
% Eduardo: Na verdade essa parte nem existe eu que esqueci de tirar (tirei agora).

Foi de extrema importância utilizar os materiais citados, pois eu entrei como bolsista
substituto, o que implica que além do tabalho a realizar, ainda havia o treinamento
que reduziria significativamente o tempo. Materiais como Github e MusiContour foram de 
grande importância. O github para a organização das tarefas diárias, e o MusiContour que 
em 1 segundo operações com matrizes que levariam vários minutos para serem calculadas.

Durante sua pesquisa o orientador busca exemplos de operações de contornos. Como existem
muitos artigos e estes mesmo possuem entre 50 e 300 páginas, a dificuldade para encontrar
o tipo de contorno desejado atrasaria o andamento do projeto. A planilha de operações tem o
objetivo de reduzir este atraso em segundos, já que todas as operações a serem procuradas
estão mapeadas e organizadas nos seguintes tópicos: Artigo, tipo de operação, página, obra,
compositor, texto ou gráfico, parágrafo ou figura. É procurar uma palavra em um dicionário.

A pesquisa me possibilitou um contato direto com diversos artigos que abordavam
o tema contornos. Este tipo de contato que não se encontra no curso de graduação,
me trouxe um conhecimento aprofundado sobre a teoria de relações de contornos muscais.
Me possiblitou também, um aprimoramento na leitura da lígua inglesa, escrita e conscisão
de resenhas, e um conhecimento sobre diversas ferramentas computacionais.

% Marcos: refira-se à etapa que você listou. Prefiro que chame de
% 'teste de operações'. Use a primeira pessoa no singular. Você pode
% dizer estas inconsistências terão posterior análise, pois a
% verificação de seu impacto na Teoria requer tempo
Durante a revisão de cada operação em particular, encontrei algumas
das operações com inconsistência no seu resultado. Estas inconsistências 
Terão posterior análise, pois a verificação de seu impacto na teoria 
requer tempo.

% Marcos: A história não aconteceu bem assim. Você encontrou as
% inconsistências depois de ter iniciado os testes. Você pode
% argumentar que fazer esses testes era mais importante que criar o
% banco de dados de figuras ou exemplos para o tutorial online. Você
% pode mostrar que a fazer esses testes foi tão importante que revelou
% inconsistências nas operações.
% Eduardo: Ok 

% Marcos: Talvez o melhor para essa parte seja começar dizendo que o
% tempo gasto inicialmente para o seu treinamento nos fez repensar as
% atividades posteriores ao mapeamento de operações.
% Eduardo: Ok
Entrei como bolsista substituto em janeiro. Isso resultou na necessidade
de um treinamento que consumiu um certo tempo. A partir desta necessidade 
houve um remanejamento no plano de trabalho. A importância de verificar cada operação 
da planilha era maior do que criar o banco de figuras ou os exemplos para o 
tutorial online, importância essa que nos mostrou que em algumas operações 
haviam inconsistências.

%%%%%%%%% bibliografia %%%%%%%%%%%%%%%%
% insere bibliografia

\renewcommand{\refname}{Referências bibliográficas (máximo 15)}
\info{Relação itemizada das referências que subsidiam a proposta de
  pesquisa, colocando as mais importantes.}

\nocite{
  Friedmann1985,
  Friedmann1987,
  Morris1987,
  Marvin1988,
  Polansky1992,
  Morris1993,
  Clifford1995,
  Quinn1997,
  Beard2003,
  Sampaio2008,
  Schultz2008,
  Schultz2009,
  Bor2009
}

\bibliographystyle{plain}
\bibliography{bibliography}

\Section{Participação em reuniões científicas e publicações}
\info{Relacionar as reuniões científicas e os títulos dos trabalhos
apresentados pelo estudante durante a vigência da bolsa. Incluir
títulos de publicações que resultaram ou se beneficiaram de seu
trabalho.}

\Section{Anexos}
\info{Anexar os resumos ou trabalhos que foram apresentados pelo bolsista
durante a vigência da bolsa.}

\end{document}
