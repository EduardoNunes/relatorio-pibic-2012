%% Local IspellDict: brasileiro

\documentclass[11pt]{article}
\usepackage{relatorio-pibic}
\usepackage[utf8x]{inputenc}
\usepackage{cite}

% usar para palavras estrangeiras: \eng{english word}
\newcommand{\eng}[1]{\textit{#1}}

\begin{document}

\graphicspath{{figs/}}

\cabecalho{Relatório Final}

\dadosRelatorioFinal
{Levantamento de operações de teorias de contornos em análises de
  obras musicais}
{Verificando inconsistências nas teorias de contornos musicais através
  de ferramentas computacionais }
{Eduardo Lago Nunes}
{Marcos da Silva Sampaio}
{Genos}
{Teoria de Relações de Contornos Musicais, Teoria Musical, Computação Musical}
{JANEIRO A JULHO DE 2012}

\resumo{Inserir texto do resumo}

\newpage

\setcounter{page}{1}
\onehalfspace

\Section{Introdução}
\label{sec:introducao}
\info{Delimitação do problema trabalhado e as conexões entre o plano
  de trabalho do bolsista e o projeto do orientador. Objetivos e
  justificativa do plano em termos de relevância para a pesquisa
  cientifica e do estado da arte.}
% importância de contornos para música, da teoria de contornos, a
% inconsistência...

% Eduardo: eu poderia colocar uma citação sua conforme eu fiz? Sua
% explicação tá mais fácil de ser entendida que a de Marvin.
% E também poderia adicionar imagem em alumas partes como essa.

% Marcos: Você pode fazer a citação sim. No entanto você copiou o
% texto tal como está em minha tese. Em um caso desse teria que vir
% entre aspas. No entanto não é bom começar uma seção com uma
% citação. Sugiro fazer escolher uma dessas opções:
% 1. reescrever esta minha definição com suas palavras para tirar as
% aspas.
% 2. inserir um parágrafo antes dessa citação. A única coisa que acho
% que poderia vir antes é um parágrafo sobre o que trata o projeto
% geral da pesquisa.
% A primeira opção é mais difícil, mas é algo que o tempo inteiro
% estamos fazendo. Vale a pena tentar.
% Eduardo: Não ficou tão completo quanto o seu, mas acho que dá pra entender.
Contorno é uma linha que marca externamente uma figura ou um objeto, uma espécie
de perímetro, caminho que cerca o redor de algum lugar ou algo. Em música pode-se
visualizar o contorno de diversos aspectos como altura, densidade, ritmo, timbre, etc.
\cite[p. 01]{Sampaio2008}.

% Eduardo: Mas marvin não cita no dele. Eu poderia colocar como está
% em baixo?
% Marcos: Pode ser assim. Apenas pode ser melhor expressado. Marvin
% fala que há estudos que mostram que contornos são mais fáceis de
% perceber do que alturas
% Eduardo: Tentei facilitar o entendimento dessa parte, vê se ficou bom.
Estudos provam que contornos são importantes pelo fato de facilitar
a percepção do ouvinte. Um contorno é mais fácil de ser reconhecido
do que alturas de notas até mesmo para ouvintes destreinados.
\cite[p. 225]{Marvin1987}.

% Marcos: quais pesquisadores? Da mesma forma, se mencionou tem que
% citar
% Eduardo: Se eu colocar como está precisa mencionar algo?
% Marcos: Sim. Quem disse isso? Se foi você, precisa mostrar como
% é. Veja no capítulo 2 da minha tese como eu lidei com isso.
A teoria de contornos visa o estudo do padrão e comportamento destes contornos.

% Marcos: quem vai desenvolver? Que pesquisas são essas? É o projeto
% do PIBIC? É outra coisa?
% Eduardo: Coloquei estas questões integrada ao parágrafo seguinte.

% Marcos: que estudos? que múltiplas teorias são essas? você falou em
% uma teoria só no parágrafo anterior.
% Eduardo: Este comentário aqui poderia ser enquadrado no parágrafo seguinte
% juntamente com o comentário de cima?

% Marcos: eu encontrei a inconsistência ao programar o
% MusiContour. Veja o projeto que eu submeti ao PIBIC no ano passado
% para ter alguma ideia nessa parte.
% Eduardo: Qual é o projeto?
% Marcos: Veja no dropbox: pibic-projeto-contornos-2011.pdf

% Marcos: acrescente em um parágrafo curto o que o projeto desta
% pesquisa propõe e como seu plano de trabalho se encaixa neste
% objetivo.
% Eduardo: Acrescentei o parágrafo que estava em discussões.

% Marcos: é melhor falar mencionar parte dessa informação na
% introdução. tem um comentário lá
% Eduardo: Coloquei na introdução.

% Marcos: O "acabou" não acrescenta nada à frase. pode falar 'o orientador
% encontrou'.
% Eduardo: Ok
Ao criar e testar software MusiCountour, o orientador encontrou
uma inconsistência na teoria dos contornos.
% Marcos: Aqui você pode dizer que eu precisava verificar o impacto da
% inconsistência nos artigos verificando cada uma das operações
% realizadas nos artigos.
% Eduardo: Ok.
A partir do encontro desta inconsistência era necessário verificar o 
impacto dela nos artigos. Para isso teria que ser feita a análise de cada
operação realizada nos artigos.
% Marcos: não são muitos artigos. Menos de 20. Melhor você falar da
% quantidade de operações que você levantou.
% Eduardo: Ok. Foram 508, coloco o número exato ou deixo como coloquei?
Foram encontradas nos artigos mais de quinhentas operações. Automatizar 
as tarefas trabalhosas economizaria tempo e energia que poderiam ser 
melhor aproveitados.
% Marcos: não fizemos automatização. O objetivo foi o mapa para
% consulta rápida de operações.
% Eduardo Ok.
O objetivo do trabalho foi criar um mapeamento destas operações para
consultas rápidas. 

% nas seções seguintes descrever as coisas
\Section{Materiais e métodos}
\label{sec:materiais}
\info{Descrição da maneira como foram desenvolvidas as atividades para
  se chegar aos objetivos propostos. Indicar o material e métodos que
  foram usados.}

Para o desenvolvimento do mapeamento de operações usamos os seguintes
materias.

% Marcos: Siga o modelo. Descreva cada item em uma ou duas linhas
% Eduardo: Ok

\begin{enumerate}
\item Planilha de mapeamento de operações de contornos. Esta planilha
  contém campos [preencher]. Vide mais informações na seção de
  resultados.
\item Mendeley\footnote{Disponível em
    \url{http://www.mendeley.com/}}. Repositório colaborativo dos
  textos da pesquisa. Este programa foi necessário para o
  compartilhamento da bibliografia.
\item MusiContour\footnote{Disponível em
    \url{http://genosmus.com/MusiContour}.}. Software de processamento
  de operações de contornos desenvolvido pelo orientador. Esta
  ferramenta foi necessária para realizar testes de operações de
  contornos.
\item Python\footnote{Disponível em
  \url{http://www.python.org/getit/}.}. Linguagem de programação de
  alto nível. Programa onde foi desenvolvido o MusiContour.
\item Linux\footnote{Disponível em
  \url{http://www.ubuntu.com/download/desktop}.}. Sistema operacional 
  de código aberto. Sistema utilizado para instalação dos softwares a 
  serem utilizados.
\item Git\footnote{Disponível em
  \url{http://git-scm.com/downloads}.}. Sistema de controle de versão
distribuído com ênfase em velocidade. Ferramenta utilizada para organização
e agilidade do projeto.
\item Github\footnote{Disponível em
  \url{https://github.com/}.}. Serviço de Web Hosting Compartilhado
para projetos que usam o controle de versionamento. Local onde organizamos
as tarefas diárias.
\item Kile e Latex\footnote{Disponível em
  \url{http://kile.sourceforge.net/}.}. Editor de código
TeX/LateX. Latex, conjunto de macros para o processador de textos. Software 
onde desvolvemos o relatório.
\end{enumerate}

% Marcos: Formate o texto abaixo como uma lista, como indico no
% comentário acima.
% Ok

 Para realização do trabalho foram cumpridas as seguintes etapas.

\begin{enumerate}
\item Familiarização com o objeto da pesquisa de 01/01/2012 à 30/01/2012. 
Consistiu na leitura da literatura sobre contornos, produção de resenhas 
e sessões de dúvidas com o orientador. 
\item Alimentação da planilha de operações, de 23/01/2012 à 06/06/2012. 
Mapeamento de todas as operações contidas nos artigos. 
\item Treinamento com ferramentas Ubuntu, Python e Musicontour, de 20/06/2012 à 11/07/2012. 
Ferramentas utilizadas para auxiliarem no trabalho. 
\item Revisão das operações mapeadas, de 28/06/2012 à 04/07/2012. 
Revisão dos cálculos de todas as operações que foram mapeadas na planilha.
\end{enumerate}

% Marcos. Estou em dúvida sobre essa parte. Vamos conversar sobre ela
% na reunião
% Eduardo: Beleza
O mapa de operações é um catálogo onde estão mapeadas todas as
operações musicais listadas nos artigos estudados.

Havia uma planilha com algumas operações já inclusas, porém ela
precisava ser revisada e formatada.
% Marcos. De que forma você analisou os artigos? Você na verdade
% extraiu as operações.
% Eduardo: Ok
% Marcos. Prefira sempre a voz ativa e a ordem direta: Eu procurei
% operações de contornos relacionadas com obras da literatura.....
% Eduardo: Ok
Eu procurei operações de contornos relacionadas com obras da literatura
musical e adicionei ao mapa de operações.
% Marcos: sugiro algo como "a planilha tem recursos como classificar,
% útil para buscar as operações por tipo..."
Concluído esta etapa, analizei cada operação inclusa na
planilha separadamente. Como haviam muitas operações e algumas
necessitavam de cáculos extensos e demorados, utilizei o software
MusiContour. Para que este software pudesse ser utilizado foi
necessário um treinamento para Python, para o MusiContour e também a
instalação do sistema operacional Ubuntu.
% Marcos: corrigir 'programas'. Você pode falar aquilo que conversamos
% na reunião. O MusiContour não havia sido testado no Windows, sistema
% que você usava, e nós não tínhamos muito tempo para testar. Vale a
% pena falar como foi mais simples você aprender a usar Ubuntu do que
% testarmos o MusiContour no Windows.
% Eduardo: Ok
Utilizei o Ubuntu para agilizar a instalação e utilização dos proramas
necessários. O MusiCountour não havia sido testado no windows, sistema 
que eu usava, e nós não tinhamos tempo para testá-lo, então optamos por 
utilizar o Ubuntu. Foi vantajoso, pois apreder a utilizar o software
era mais prático.
% Marcos: colocar Kile com inicial em maiúscula. É muito mais
% significativo você dizer que aprendeu a usar o Latex do que o
% Kile. O Kile é só um editor. Você pode escrever código Latex até em
% um bloco de notas.
% Eduardo: Ok
Finalmente para elaborar o relatório utilizei o software Latex, que
permite uma facilidade no trabalho de edição, um aprendizado adicional
sobre programação, e um aprofundamento nos softwares citados em
materiais.

\Section{Resultados}
\label{sec:resultados}
\info{Relação dos resultados ou produtos obtidos durante a execução da
  pesquisa, indicando os avanços no conhecimento disponível obtidos
  com a execução da pesquisa.}

% planilha com mapeamento de operações e o que funciona
% aprendizado

% Eduardo: Ok. Consegui distinguir a diferença entre mapeamento e planilha a
% partir da explicação do e-mail.

% Marcos: Maravilha!!!

% Marcos: evite muitos adjetivos no texto. Várias vezes você diz que
% algo é muito importante. No caso dos testes basta dizer o que eles
% são e o que revelam
% Eduardo : ok
O mapeamento das operações foi criado com intuito de servir como uma
espécie de catálgo de operações. Este catálogo serve para consultas
rápidas de um determinado contorno (Vide seção de discussão).
% Marcos: melhor "...testei todas as operações listadas e encontrei
% inconsistências..."
Após o mapeamento das operações na planilha testei todas as operações
listadas e encontrei inconsistências.
% Marcos: Este é o local mais apropriado para fazer uma descrição da
% planilha. Você pode até inserir uma referência cruzada na seção
% anterior dizendo que há mais informações sobre a planilha nesta
% seção
% Eduardo: Quais inforamções eu deveria colocar? eu já citei a planilha 
% toda na parte discussão.

Este trabalho teve como resultados secundários:

\begin{enumerate}
\item Aprimoramento da leitura em língua inglesa.
\item Aprofundamento do conhecimento sobre contornos.
\item Aprendizado sobre Ubuntu, Python, Kile, MusiContour.
\item Aprimoramento da escrita de textos acadêmicos.
\item Aprimoramento de organização e técnicas de estudo.
\end{enumerate}

\Section{Discussão}
\label{sec:discussao}
\info{Expor de modo sucinto a contribuição do seu projeto ao projeto
  de pesquisa do orientador e ao conhecimento científico da sua área,
  apresentando as implicações para futuros trabalhos que podem ser
  desenvolvidos.}
% comentar o que foi dito nas outras seções. ir além da descrição

% Marcos: você pode dizer que as ferramentas que utilizou deu
% agilidade ao trabalho e economizou tempo e energia. Com o GitHub
% você manteve sua lista de tarefas organizada. Vale a pena colocar um
% link para a sua lista de tarefas. Essa é a hora de impressionar seu
% leitor (veja o link mais abaixo). Sobre o MusiContour você pode
% dizer que economizou muito tempo e mostrar como se faz uma matriz de
% comparação. Veja meu texto na dissertação para ter uma ideia de como
% descrever isso. É bom você mostrar a dificuldade para as pessoas
% entenderem como fazer isso gasta tempo.

% https://github.com/msampaio/MusiContour/issues/assigned/EduardoNunes?direction=desc&sort=created&state=closed

Foi necessário utilizar os materiais citados, eu entrei como bolsista
substituto, o que implica que além do tabalho a realizar, ainda havia
o treinamento que reduziria o tempo de trabalho. Materiais como Github
e MusiContour foram utilizados para suprir este tempo que foi
reduzido. O github para a organização das tarefas diárias\footnote{As
  tarefas realizadas estão disponíveis para consulta em
  \url{http://goo.gl/4Ie7c}.}, e o MusiContour que em 1 segundo
operações com matrizes que levariam minutos para serem calculadas.

Cálculo que o Musicontour faz:
% Eduardo: Seria bom colocar um print nessa parte
% Marcos: Vamos ver essa parte na reunião

% Marcos: A organização em tópicos deve estar associada à parte da
% discussão do resultado. É preciso colocar um trecho da planilha em
% anexo. Eu vou cuidar do anexo.
% Eduardo: Certo. Mas vamos assciar o texto ao anexo, ou o anexo vai ser
% associado ao texto?
% Marcos: Não entendi sua pergunta

O mapeamento de operações tem o objetivo de reduzir este atraso em
segundos, já que todas as operações a serem procuradas estão mapeadas
e organizadas nos seguintes tópicos: Artigo, tipo de operação, página,
obra, compositor, texto ou gráfico, parágrafo ou figura.

A pesquisa me possibilitou um contato direto com diversos artigos que abordavam
o tema contornos. Este tipo de contato que não se encontra no curso de graduação,
me trouxe um conhecimento aprofundado sobre a teoria de relações de contornos muscais.
Me possiblitou também, um aprimoramento na leitura da lígua inglesa, escrita e conscisão
de resenhas, e um conhecimento sobre diversas ferramentas computacionais.

% Marcos: depois de 'no seu resultado' coloque '(vide lista abaixo)'
% Eduardo: Ok.
Durante o teste de operações, encontrei algumas das operações com
inconsistência no seu resultado (vide lista abaixo). Estas inconsistências terão posteriores
análises, pois a verificação de seu impacto na teoria requer tempo.
% Eduardo, eu teria que explicar algo sobre as operações citadas??
% Coloque o nome da operação por extenso e a referência bibliográfica
% como fizemos noa introdução: 
% Eduardo: Ok

% Marcos: pode remover o 'operações com inconsistência'
% Eduardo: Ok
\begin{enumerate}
\item Pg 241/241, Operação Countour Class Vector II \cite[p. 241]{Friedmann1985}.
\item Pg 242, Operação Contour Similarity. \cite[p. 241]{Quinn1997}.
\item Pg 262, Contour + Similarity. \cite[p. 241]{Quinn1997}.
\item 113, Operação Contour Class. \cite[p. 241]{Schultz2008}.
\end{enumerate}

% Marcos: melhor colocar a data exata: 02/01/2012
%Eduardo: Ok
Entrei como bolsista substituto em 02/01/2012. Isso resultou na
necessidade de um treinamento que consumiu um certo tempo. A partir
desta necessidade houve um remanejamento no plano de trabalho.
% Marcos: melhor: 'A verificação de operações da planilha é mais
% importante do que...'
A verificação de operações da planilha é mais importante que
criar o banco de figuras ou os exemplos para o tutorial online,
importância essa que nos mostrou que em algumas operações haviam
inconsistências.

%%%%%%%%% bibliografia %%%%%%%%%%%%%%%%
% insere bibliografia

\renewcommand{\refname}{Referências bibliográficas (máximo 15)}
\info{Relação itemizada das referências que subsidiam a proposta de
  pesquisa, colocando as mais importantes.}

\nocite{
  Friedmann1985,
  Friedmann1987,
  Morris1987,
  Marvin1987,
  Marvin1988,
  Polansky1992,
  Morris1993,
  Clifford1995,
  Quinn1997,
  Beard2003,
  Sampaio2008,
  Schultz2008,
  Schultz2009,
  Bor2009
}

\bibliographystyle{plain}
\bibliography{bibliography}

\Section{Participação em reuniões científicas e publicações}
\info{Relacionar as reuniões científicas e os títulos dos trabalhos
apresentados pelo estudante durante a vigência da bolsa. Incluir
títulos de publicações que resultaram ou se beneficiaram de seu
trabalho.}

\Section{Anexos}
\info{Anexar os resumos ou trabalhos que foram apresentados pelo bolsista
durante a vigência da bolsa.}

\end{document}
