\documentclass[11pt]{article}
\usepackage{relatorio-pibic}
\newcommand{\eng}[1]{\textit{#1}}
\usepackage[utf8x]{inputenc}

\begin{document}

\graphicspath{{figs/}}

\cabecalho{Relatório Final}

\dadosRelatorioFinal
{Levantamento de operações de teorias de contornos em análises de
  obras musicais}
{Verificando inconsistências nas teorias de contornos musicais através
  de ferramentas computacionais }
{Eduardo Lago Nunes}
{Marcos da Silva Sampaio}
{Genos}
{Inserir 3 palavras chave}
{JANEIRO A JULHO DE 2012}

\resumo{Inserir texto do resumo}

\newpage

\setcounter{page}{1}
\onehalfspace

\section{Introdução}
\label{sec:introducao}
% importância de contornos para música, da teoria de contornos, a inconsistência...


% nas seções seguintes descrever as coisas
\section{Materiais e métodos}
\label{sec:materiais}
% materiais: artigos, MusiContour (dizer que faz parte da pesquisa), planilha
% métodos: como fez cada coisa

MusiContour\footnote{Disponível em \url{http://genosmus.com/MusiContour}.}

\section{Resultados}
\label{sec:resultados}
% planilha com mapeamento de operações e o que funciona
% aprendizado

\section{Discussão}
\label{sec:discussao}
% comentar o que foi dito nas outras seções. ir além da descrição

\end{document}
