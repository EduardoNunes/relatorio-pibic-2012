%% Local IspellDict: brasileiro

\documentclass[11pt]{article}
\usepackage{relatorio-pibic}
\usepackage[utf8x]{inputenc}
\usepackage{cite}

% usar para palavras estrangeiras: \eng{english word}
\newcommand{\eng}[1]{\textit{#1}}

\begin{document}

\graphicspath{{figs/}}

\cabecalho{Relatório Final}

\dadosRelatorioFinal
{Levantamento de operações de teorias de contornos em análises de
  obras musicais}
{Verificando inconsistências nas teorias de contornos musicais através
  de ferramentas computacionais }
{Eduardo Lago Nunes}
{Marcos da Silva Sampaio}
{Genos}
% Eduardo Inserir 3 palavras chave
{Teoria de Relações de Contornos Musicais, Teoria Musical, Computação Musical}
{JANEIRO A JULHO DE 2012}

\resumo{Inserir texto do resumo}

\newpage

\setcounter{page}{1}
\onehalfspace

% Marcos: orientações gerais:

% Inicialmente não se preocupe tanto com o texto, mas com os dados
% para usar no texto. Depois que tiver a lista dos materiais, das
% etapas cumpridas, dos resultados, das dificuldades, você pode
% refletir sobre cada um e escrever anotações sobre eles. Em seguida
% você vai transformando essas anotações no seu texto

% Foque primeiro nos materiais e métodos. O que você fez, como você
% fez. Arrume esta seção antes das outras

% O seu leitor pode não entender nada de contornos. Com o seu texto
% ela precisa entender que havia um problema para ser
% resolvido---simplificar os testes e verificação de operações de
% contornos---e você resolveu com a planilha

% Não use vírgula entre sujeito e verbo

\Section{Introdução}
\label{sec:introducao}
\info{Delimitação do problema trabalhado e as conexões entre o plano
  de trabalho do bolsista e o projeto do orientador. Objetivos e
  justificativa do plano em termos de relevância para a pesquisa
  cientifica e do estado da arte.}
% importância de contornos para música, da teoria de contornos, a
% inconsistência...

% Marcos: O que são contornos? Você sempre pode parafrasear alguém e
% citar.
Contornos são importantes pelo fato de fazer ser mais perceptível uma
certa melodia ou ritmo que esteja sendo executado, até mesmo para
pessoas com ouvido destreinado.

% Marcos: Qual a sua ideia neste parágrafo? Qual ferramenta? De quem é
% a teoria?

% Marcos: Remover vírgula entre sujeito e verbo
A teoria de contornos visa estudar esta ferramenta com intuito de
desenvolver técnicas para análises do estudo dos padrões e
comportamento destes contornos.
% Marcos: quem vai desenvolver? Que pesquisas são essas? É o projeto
% do PIBIC? É outra coisa?
A partir destes estudos, desenvolver uma solução para uma
inconsistência encontrada durante as pesquisas do orientador.

% Eduardo: passei essa parte de comentário pra cima por se tratar da questão anterior
% Marcos: pense no objetivo geral do seu plano de trabalho para ter
% mais facilidade de escrever este trecho: "Levantamento de operações
% de teorias de contornos em análises de obras musicais". A planilha é
% um recurso que você usou para fazer este levantamento de
% operações.

% nas seções seguintes descrever as coisas
\Section{Materiais e métodos}
\label{sec:materiais}
\info{Descrição da maneira como foram desenvolvidas as atividades para
  se chegar aos objetivos propostos. Indicar o material e métodos que
  foram usados.}
% Marcos: Não precisa dizer que não havia tempo hábil. Basta dizer que
% você realizou o trabalho usando estes materiais.

Para o desenvolvimento da planilha de operações houve a necessidade de
utilizarmos diversos materias.

% Marcos: Pensei melhor e acho melhor você fazer uma lista com uma
% linha de explicação para cada um:
% mendeley \footnote{link}. Repositório colaborativo de artigos da
% pesquisa
% MusiContour \footnote{link}. Software de processamento de operações
% de contornos.....
%Eduardo, dá uma conferida nesta parte, pra ver se está como orientado acima.

Materiais:
Planilha de operações.
Mendeley\footnote{Disponível em
  \url{http://www.mendeley.com/}.}. Mendeley, local onde o grupo de pesquisa
armazena os arquivos que foram estudados. 
Artigos\footnote{Disponível em
  \url{http://www.mendeley.com/groups/1138861/_/papers/}.}. Material da pesquisa
que foi analizado para a extração dos contornos.
MusiContour\footnote{Disponível em \url{http://genosmus.com/MusiContour}.}. Software 
desenvolvido pelo orientador que faz os cáculos das operações de contornos.
  Python\footnote{Disponível em  \url{http://www.python.org/getit/}.}. Editor e 
linguagem de programação em que foi desenvolvido o software MusiContour.
  Linux\footnote{Disponível em
  \url{http://www.ubuntu.com/download/desktop}.}. Linux sistema operacional utilizado 
  durante a pesquisa.
  Git\footnote{Disponível em
  \url{http://git-scm.com/downloads}.}. Sistema de controle de versão distribuído com 
ênfase em velocidade.
  Github\footnote{Disponível em
  \url{https://github.com/}.}. Serviço de Web Hosting Compartilhado para projetos que 
usam o controle de versionamento e também onde organizamos
as tarefas diárias.
  Kile e Latex\footnote{Disponível em
  \url{http://kile.sourceforge.net/}.}. Editor de código TeX/LateX. Latex, conjunto de macros 
  para o processador de textos.
  Por fim, a planilha de operações que é o principal material e também o resultado do trabalho.

% Marcos. Remova o 'foi necessário' e apenas diga que para realizar o trabalho
% você cumpriu as etapas listadas
%Eduardo: Removido.
Para realização do trabalho foram cumpridas todas as etapas listadas.

% Marcos. Coloque uma ou duas linhas em cada item para explicar o que é.
\begin{enumerate}
% Marcos. Você pode dizer que o treinamento consistiu na leitura da
% literatura sobre contornos, produção de resenhas e sessões de
% dúvidas com o orientador.
% Eduardo: OK
\item Treinamento. Consistiu na leitura da literatura sobre contornos, produção
de resenhas e sessões de dúvidas com o orientador.
\item Leitura dos artigos e elaboração de resenhas, para um melhor entendimento 
do trabalho.
% Marcos. Alimentação da planilha de operações fica melhor
% Eduardo: ok
\item Alimentação da planilha de operações a partir dos artigos analizados.
% Marcos. Junte instalação de ubuntu com treinamento em python e musicontour e
% chame de instalação e treinamento em ferramentas.
% Eduardo: Ok
\item Treinamento em ferramentas Ubuntu, Python e Musicontour, para auxiliar
no trabalho.
\item Revisão das operações contidas na planilha
% Marcos. Remova a elaboração do relatório. Em um projeto acho
% necessário ter esse item, mas no próprio relatório não.
% Eduardo: Removido.
\end{enumerate}

% Marcos. Como falei, coloque essas informações direto na lista
% Eduardo: Ok


 %Mova as dificuldades encontradas para a
% parte de discussão
% Eduardo: Movido


% Marcos. Neste local apenas informe no que consiste a planilha. Tente
% não passar de duas linhas. Fale sobre a necessidade de ser revisada
% e formatada na parte de discussão. Siga esse princípio para as
% outras partes
% Eduardo: Dá uma olhada pra ver se ficou beleza.
A planilha de operações é um catálogo onde estão mapeadas todas as operações musicais listadas
nos artigos estudados.

% Marcos. Você usa a expressão contexto histórico-musical, mas não
% define o que isso é. Em casos assim você primeiro precisa definir a
% expressão para depois usar. De qualquer forma eu não concordo com
% essa expressão. Basta dizer que você mapeou as operações de acordo
% com a referência de origem.
% Eduardo: Corrigido (Esta parte fica grande assim mesmo?).
Havia uma planilha com algumas operações já inclusas, porém ela precisava ser revisada e estilizada.
Todos os artigos foram analizados e todas as operações mapeadas de acordo com a referência de origem.
Concluído esta etapa, bastava analizar cada operação inclusa na planilha separadamente. Como haviam 
muitas operações e algumas necessitavam de cáculos extensos e demorados, foi utilizado o software MusiContour. 
Para que este software pudesse ser utilizado era necessário um treinamento para Python, para o MusiContour 
e também a instalação do sistema operacional Ubuntu. O Ubuntu foi utilizado para agilizar a instalação e 
utilização dos proramas necessários. Como eu utilizava Windows estávamos perdendo tempo tentando descobrir 
comandos de instalações e traduções de linguagem do Ubuntu para Windows.
Finalmente para elaborar o relatório utilizei o software kile, que permite uma facilidade no trabalho
de ediçã, um aprendizado adicional sobre programação, e um aprofundamento nos softwares citados em materiais.

\Section{Resultados}
\label{sec:resultados}
\info{Relação dos resultados ou produtos obtidos durante a execução da
  pesquisa, indicando os avanços no conhecimento disponível obtidos
  com a execução da pesquisa.}

% planilha com mapeamento de operações e o que funciona
% aprendizado

% Marcos: coloque esse comentário na parte de discussão. Aqui você
% apenas diz quais os resultados.
% Eduardo: Colocado.


% Marcos. Não diga aqui que foi cumprido no prazo. Isso dilui o
% objetivo mais importante, de entender o resultado.
% Eduardo: tirei a parte que fala o que é a plnilha, pois falei no quesito anterior.
A planilha de operações de contornos foi o principal objetivo do tabalho. 
% Marcos. Antes desta lista diga que este trabalho também teve
% resultados secundários. Coloque todos menos a planilha
% Eduardo: dá uma olhadinha.
Além do objetivo de concluir a planilha de operações, obtivemos alguns resultados secundários.

\begin{enumerate}
% Marcos. aprimoramento da leitura em língua inglesa
% Eduardo: ok
\item Aprimoramento da leitura em língua inglesa.
% Marcos. padronize a forma de falar. Aprofundamento do conhecimento
% sobre contornos.
% Eduardo: Ok
\item Aprofundamento do conhecimento sobre contornos.
\item Aprendizado sobre Ubuntu, Python, Kile, MusiContour.
% Marcos. Aprimoramento da escrita de textos acadêmicos
% Eduardo: Ok
\item Aprimoramento da escrita de textos acadêmicos.
% Marcos. Aprimoramento de organização e técnicas de estudo
% Eduardo: Ok
\item Aprimoramento de organização e técnicas de estudo.
\end{enumerate}


\Section{Discussão}
\label{sec:discussao}
\info{Expor de modo sucinto a contribuição do seu projeto ao projeto
  de pesquisa do orientador e ao conhecimento científico da sua área,
  apresentando as implicações para futuros trabalhos que podem ser
  desenvolvidos.}
  

  Neste passo foram encontradas dificuldades como a língua inglesa (todos
os artigos estão escritos em inglês), e o entendimento de algumas operações. Estas dificuldades
foram esclarecidas com a leitura dos textos, eplicações em reuniões e elaboração de resenhas.

% comentar o que foi dito nas outras seções. ir além da descrição
% EDUARDO não mexi nessa parte ainda.
Foi de extrema importância utilizar os materiais citados, pois eu entrei como bolsista 
substituto, o que implica que além do tabalho a realizar, ainda havia o treinamento
que reduziria significativamente o tempo. Materiais como Github que foi de grande importância 
para a organização das tarefas diárias, ou MusiContour que executa em 1 segundo operações
com matrizes que levariam vários minutos para serem calculadas.

Durante sua pesquisa o orientador busca exemplos de operações de contornos. Como existem
muitos artigos e estes mesmo possuem entre 50 e 300 páginas, a dificuldade para encontrar
o tipo de contorno desejado atrasaria o andamento do projeto. A planilha de operações tem o
objetivo de reduzir este atraso em segundos, já que todas as operações a serem procuradas
estão mapeadas e organizadas nos seguintes tópicos: Artigo, tipo de operação, página, obra, 
compositor, texto ou gráfico, parágrafo ou figura. É procurar uma palavra em um dicionário.

A pesquisa me possibilitou um contato direto com diversos artigos que abordavam
o tema contornos. Este tipo de contato que não se encontra no curso de graduação,
me trouxe um conhecimento aprofundado sobre a teoria de relações de contornos muscais.
Me possiblitou também, um aprimoramento na leitura da lígua inglesa, escrita e conscisão
de resenhas, e um conhecimento sobre diversas ferramentas computacionais.

% Marcos: Prefira sempre a ordem direta das frases e evite a voz
% passiva. Diga que você encontrou incoerências após testar as
% operações. Substitua o "devidas providências" por algo como
% "posterior avaliação"
% Eduardo: Apaguei aquele parágrafo e o refiz.
Durante a revisão de cada operação em particular, encontramos algumas com inconsistência
no seu resultado. Estas inconsistências nos levam a um tipo de análise ainda mais aprofundada,
devemos avaliar se o impacto das mesmas causará algum impacto significativo para a teoria dos contornos.

Houve um pequeno desvio no plano de trabalho. A parte principal, que era a planilha de operações, 
já estava pronta. Para a nossa surpresa, nos deparamos com algo de muita importância que foram 
as inconsistências, este foi mais um motivo para demonstrar a importância de ter sido realizada a construção 
da planilha de operações. Contudo, eu entrei como bolsista substituto em Janeiro. O que necessitou de um 
treinamento que levou um certo tempo e nos obrigou a modificar o programa original.



% Marcos: Não coloque a bibliografia por extenso aqui. Eu irei
% formatar a bibliografia automaticamente depois. Se quiser
% acrescentar algum livro, apenas coloque o nome do autor e o ano como
% um comentário, dessa forma:

% Morris 1993

%%%%%%%%% bibliografia %%%%%%%%%%%%%%%%
% insere bibliografia

\renewcommand{\refname}{Referências bibliográficas (máximo 15)}
\info{Relação itemizada das referências que subsidiam a proposta de
  pesquisa, colocando as mais importantes.}

\nocite{
  Friedmann1985,
  Friedmann1987,
  Morris1987,
  Marvin1988,
  Polansky1992,
  Morris1993,
  Clifford1995,
  Quinn1997,
  Beard2003,
  Sampaio2008,
  Schultz2008,
  Schultz2009,
  Bor2009
}

\bibliographystyle{plain}
\bibliography{bibliography}

\Section{Participação em reuniões científicas e publicações}
\info{Relacionar as reuniões científicas e os títulos dos trabalhos
apresentados pelo estudante durante a vigência da bolsa. Incluir
títulos de publicações que resultaram ou se beneficiaram de seu
trabalho.}

\Section{Anexos}
\info{Anexar os resumos ou trabalhos que foram apresentados pelo bolsista
durante a vigência da bolsa.}

\end{document}
